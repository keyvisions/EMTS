\chapter{Standard Model}
\label{ch:standard-model}

Building on the EMTS framework (Chapter~\ref{ch:emts-framework}), we map Standard Model particles and forces onto the complex plane. In quantum theory, the state encodes probabilities.\cite{griffiths_qm} In this framework, oscillations with frequency $\omega$ link to detection likelihoods. Particle vs.~wave is reconciled: coherent evolution (\emph{wave}) sets channel rates, while projection events (\emph{particles}) are localized outcomes.

\section*{Particle vs. Wave}
\begin{itemize}
  \item \textbf{State vector}: carries phase and interferes.
  \item \textbf{Measurement}: projects to outcomes with Born probabilities.
  \item \textbf{Field view}: particles are quantized excitations; waves are coherent field amplitudes.
\end{itemize}

With $z(t)=r e^{i\omega t}$ and windows $W_f(\theta)$ from Chapter~\ref{ch:fundamental-forces}, a minimal rate model is
\[
R_f(r)=\frac{\omega}{2\pi}\int_0^{2\pi} W_f(\theta)\,g(\theta,r)\,d\theta.
\]

\section*{Quarks in the complex plane}
In the EMTS picture we write a generic excitation as $\EMTSz$, with $r$ encoding a space-like scale and $\phaseTime$ encoding time-like phase. Quarks can be viewed as excitations whose complex-plane behaviour is constrained by threefold structure:
\begin{itemize}
  \item \textbf{Colour charge} corresponds to how the excitation winds in an internal copy of the complex plane, with three preferred phase orientations (``red, green, blue'') in an $SU(3)$-like subspace.
  \item \textbf{Confinement} arises because a single quark's complex trajectory cannot close in the observable $m$--$E$ plane; only colour-neutral combinations lead to closed orbits and stable projections.
  \item \textbf{Flavour} (up, down, strange, etc.) is tied to distinct radii $r$ and characteristic angular frequencies $\omega$, leading to different effective masses and couplings.
\end{itemize}
In this view, quarks occupy specific bands in $r$ and families of allowed phase patterns in $\theta$, with hadrons emerging as composite closed paths whose joint projection appears as a single particle.

\section*{Leptons as simpler orbits}
Leptons lack colour charge and so correspond to simpler trajectories on the same complex plane. Their key properties emerge from how their paths relate to the real and imaginary axes:
\begin{itemize}
  \item \textbf{Charged leptons} (electron, muon, tau) follow orbits with a nonzero average real projection $\langle m_z \rangle$, giving rest mass, while their interaction with the electromagnetic field shifts $\theta$ and modulates $E$.
  \item \textbf{Neutrinos} are nearly lightlike: their trajectories lie close to the imaginary axis, with very small $m$ but nontrivial phase evolution, which can support flavour oscillations as slow precessions between nearby orbits.
  \item \textbf{Generations} correspond to nested shells in $r$ with similar angular patterns but different characteristic frequencies, reflecting the mass hierarchy without changing the basic complex geometry.
\end{itemize}
Leptons thus occupy cleaner, less composite regions of the complex plane, making them ideal probes of how $m$ and $E$ trade off along a single worldline.

\section*{Forces as phase symmetries}
Forces in the Standard Model act by reshaping or constraining motion on the complex plane rather than by pushing in ordinary space alone. We can sketch them as follows:
\begin{itemize}
  \item \textbf{Electromagnetism} tracks changes in the global phase of charged trajectories. A $U(1)$ gauge transformation is a shift $\theta \mapsto \theta + \alpha$, leaving $r$ fixed but altering interference patterns and detection rates.
  \item \textbf{Weak interactions} mix components of $z$ associated with different leptonic and quark flavours. Geometrically this can be pictured as rotations between nearby orbits in a multi-dimensional complex space, with massive gauge bosons mediating large, localized phase jumps.
  \item \textbf{Strong interactions} constrain colour phases so that only colour-neutral combinations yield closed, low-action paths. Gluon exchange continually rewires how individual quark trajectories share a joint complex-phase structure inside hadrons.
\end{itemize}
All three gauge forces can be summarised as rules about which deformations of the complex trajectory $z(t)$ leave physical rates invariant and which cost action, echoing the role of symmetries and gauge fields in the usual Standard Model.

\section*{A unified placement}
In summary, the complex plane provides a common stage:
\begin{itemize}
  \item \textbf{Quarks} occupy structured, colour-charged orbits whose composites form closed, observable paths.
  \item \textbf{Leptons} trace cleaner, single-particle trajectories distinguished mainly by radius and frequency.
  \item \textbf{Forces} appear as symmetries and constraints on allowed deformations of these complex paths.
\end{itemize}
This speculative mapping does not replace the field-theoretic Standard Model,\cite{peskin_schroeder} but it offers a geometric intuition: all fundamental matter and forces inhabit different patterns of motion and symmetry on an underlying complex $m$--$E$ plane.
