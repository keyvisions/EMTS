% Part III — Implementation
\chapter{Electron Angels and Demons: Wave Duality and Collective Modes}
\label{ch:electron-demons}

\section{Motivation: the dual nature of the electron}

The electron is simultaneously a particle and a wave.  As a particle it carries a well-defined mass $m_e$ and charge $-e$; as a wave it is described by a complex-valued wavefunction $\psi(\mathbf{x},t)=|\psi|\,e^{i\phi}$ that obeys the Schr\"odinger (or Dirac) equation.  When many-electron wavefunctions overlap, the phases $\phi_k$ can align or oppose, producing entirely new macroscopic states that bear little resemblance to the bare electron.

Within the EMTS framework the electron sits at
\[
  z_e = m_e + iE_e = r_e\,e^{i\theta_e},
\]
where $r_e = \sqrt{m_e^2+E_e^2}$ is the total mass-energy magnitude and $\theta_e = \arctan(E_e/m_e)$ records the partition between rest mass and kinetic/potential energy.  For a free electron at rest $\theta_e \approx 0$ (real axis), while a high-energy electron is rotated toward the imaginary axis.

A \emph{collective} state of $N$ electrons is then described by the vector sum
\[
  Z_{\text{coll}} = \sum_{k=1}^{N} z_k = \sum_{k=1}^{N} r_k\,e^{i\theta_k}.
\]
The character of this sum --- constructive or destructive --- determines whether the system behaves as an \textbf{Angel} or a \textbf{Demon} in the taxonomy developed below.

\section{Constructive interference: Angels}
\label{sec:angels}

\subsection{Definition}

Consider $N$ electrons whose EMTS phases are \emph{aligned}:
\[
  \theta_k \approx \bar\theta \quad \forall\, k.
\]
The collective vector is then approximately
\[
  Z_{\text{Angel}} \approx \left(\sum_{k=1}^{N} r_k\right) e^{i\bar\theta},
\]
which is a single EMTS point with modulus $\sim Nr_e$ and essentially the same argument as the individual electrons.  The real (mass-like) and imaginary (energy-like) components both scale up together.

\subsection{Physical content}

Phase alignment means that maxima of the individual wavefunctions coincide.  Some familiar realizations:
\begin{itemize}
  \item \textbf{Cooper pairs and superconductivity.}  Below $T_c$, electrons near the Fermi level pair into Cooper pairs with equal and opposite momenta and opposite spins; their pair wavefunction is a single coherent superposition.  In EMTS, the pair amplitude has its phases locked so that $Z_{\text{pair}}$ sits solidly on the real axis (dominant mass, minimal energy dispersion), explaining the gap and lossless transport.\cite{ashcroft_mermin}
  \item \textbf{Coherent many-body ground states.}  In a crystal, Bloch electrons form bands; near a filled lower band the coherent sum of occupied states gives a stable, low-energy configuration --- the background ``sea'' of condensed matter.
  \item \textbf{Stimulated emission.}  In a laser medium, photon-driven transitions reinforce phase alignment of atomic dipoles; the emitted photons are the EM analogue of the Angel state.
\end{itemize}

\subsection{EMTS geometry}

In the complex plane, an Angel state corresponds to a tight \emph{cluster} of $N$ points near a single $z_e$, their vector sum pointing radially outward with length $\sim Nr_e$.  The projection $\Re(Z_{\text{Angel}}) = Nr_e\cos\bar\theta$ is large: Angels are \emph{observable} in the real (mass/matter) sector.

\section{Destructive interference: Demons}
\label{sec:demons}

\subsection{Definition}

Now suppose the phases of the participating electrons are \emph{anti-aligned}:
\[
  \theta_k = \bar\theta + \delta_k, \quad \sum_{k=1}^{N} e^{i\delta_k} \approx 0.
\]
In the simplest two-electron toy model with $\theta_1 = \theta$ and $\theta_2 = \theta + \pi$:
\[
  Z_{\text{Demon}} = r_1 e^{i\theta} + r_2 e^{i(\theta+\pi)} = (r_1 - r_2)\,e^{i\theta}.
\]
When $r_1 = r_2$ the individual mass-energy contributions \emph{cancel exactly}: $Z_{\text{Demon}} = 0$.  Even when cancellation is only partial, the dominant real-axis component is suppressed, and what survives is primarily imaginary --- \emph{energy without mass}.

\subsection{Electrons as plasmons}

The most experimentally dramatic Demon is the \emph{plasmon} --- the collective charge-density oscillation of the electron gas.\cite{ashcroft_mermin}  A classical derivation illustrates the Demon mechanism:

\begin{enumerate}
  \item Each conduction electron in a metal has individual EMTS vector $z_k = m_e + iE_k$.
  \item A charge displacement excites a density wave $n(\mathbf{x},t)=n_0 + \delta n\,\cos(\mathbf{q}\cdot\mathbf{x}-\omega t)$.
  \item The restoring Coulomb forces couple all electrons so that individual particle identities dissolve into a collective mode at the plasma frequency
        \[
          \omega_p = \sqrt{\frac{n_0 e^2}{\epsilon_0 m_e}}.
        \]
  \item This new quasi-particle --- the \emph{plasmon} --- has an effective mass determined by $\omega_p$ and a dispersion $\omega(\mathbf{q})$ very different from the bare electron.
\end{enumerate}

In EMTS, the plasmon is described not by any single $z_k$ but by the \emph{residual} of the collective sum after destructive cancellation.  Its EMTS vector sits near the imaginary axis (large $\theta$, small real component), reflecting the fact that the plasmon has very little ``matter-like'' character --- it is primarily an \emph{energy excitation} propagating through the electron sea.

\subsection{The Demon analogy}

The metaphor of the \emph{Demon} is apt:
\begin{itemize}
  \item Like a demon, the plasmon exploits the absence of order (destructive phase cancellation) to manifest as something qualitatively different from the individual electrons.
  \item It ``hides'' behind the anti-phase interference --- invisible to direct mass measurement, yet capable of transporting energy at $\omega_p$ and mediating optical phenomena at UV frequencies.
  \item Demons are \emph{unnatural} in the sense that they require a specific cancellation to persist; they depend on the maintenance of phase opposition across the participating electrons.
\end{itemize}

\subsection{EMTS geometry}

In the complex plane, the individual electron points spread symmetrically around the origin so that their vector sum $Z_{\text{Demon}}$ is close to (or exactly at) zero.  The plasmon is then the \emph{phase-space excitation} of this near-cancellation --- a small residual that rotates rapidly near the imaginary axis, with $|\Re(Z)| \ll |\Im(Z)|$.

\section{An EMTS taxonomy of collective electron modes}
\label{sec:taxonomy}

The constructive/destructive axis is not binary.  We can organize all collective electron states on a spectrum according to the degree of phase coherence in the EMTS sum.  Four named classes arise naturally, analogous to a celestial hierarchy:

\begin{center}
\renewcommand{\arraystretch}{1.4}
\begin{tabular}{llll}
\hline
\textbf{Name} & \textbf{Phase relation} & \textbf{$|Z_{\text{coll}}|/Nr_e$} & \textbf{Physical archetype} \\
\hline
Archangel & Extreme constructive, $\delta_k\to 0$ & $\to 1$ & BEC, superfluid, lasing mode \\
Angel     & Constructive, partial alignment & $0.5 \text{ to } 1$ & Cooper pairs, band ground state \\
Demon     & Destructive, partial cancellation & $\lesssim 0.1$ & Plasmons, acoustic modes \\
Devil     & Extreme destructive, near-total cancel & $\to 0$ & Acoustic plasmon, demon quasi-particle \\
\hline
\end{tabular}
\end{center}

\subsection{Angels: the natural order}

Constructive interference is the \emph{ground-state tendency}: a system of electrons minimizes its energy by occupying coherently the lowest available modes, with wavefunctions adding positively.  Angels represent equilibrium.  Left to evolve without perturbation, an electron ensemble relaxes toward an Angelic configuration because such states minimize free energy.

\subsection{Archangels: extreme coherence}

When phase alignment is \emph{forced} across a macroscopic number of particles --- as in a Bose-Einstein condensate (BEC), a superfluid, or an optical laser --- the collective EMTS vector points almost perfectly along a single direction in the complex plane, with $|Z|\approx Nr_e$.  This is the Archangel limit:
\[
  Z_{\text{Arch}} = N r_e \, e^{i\bar\theta}, \qquad \text{all } \theta_k \equiv \bar\theta.
\]
Archangels are extraordinary: they require macroscopic phase locking, typically enforced by a symmetry-breaking mechanism (BCS gap equation, Gross-Pitaevskii condensation, stimulated emission).  Their EMTS projection on the real axis is the largest possible, making them the \emph{most observable} collective electron states.

\subsection{Demons: the plasmon family}

Standard plasmons (bulk, surface, acoustic) form the core of the Demon class.  Their salient EMTS property is:
\begin{itemize}
  \item Small $|\Re(Z_{\text{coll}})|$ --- minimal ``matter'' character.
  \item Large $|\Im(Z_{\text{coll}})|$ / fast $\theta$ rotation --- dominant energy character.
  \item Finite lifetime: phase opposition is maintained only for the coherence time $\tau_p$; scattering and damping (Landau damping) restore the Angel ground state.
\end{itemize}

The surface plasmon polariton (SPP) is a Demon that lives at the interface between a conductor and a dielectric, coupling to photons --- a mixed photonic/electronic Demon with EMTS vector straddling both electromagnetic and material quadrants.

\subsection{Devils: extreme destructive states}

The Devil limit is total cancellation $Z \to 0$.  This requires fine-tuning or protective symmetry.  Known physical candidates:

\begin{itemize}
  \item \textbf{Acoustic plasmon \& the ``electronic Demon.''} In a multi-band metallic system where two bands with nearly equal but opposite effective-mass contributions coexist, destructive cancellation of the conventional Drude weight can yield a mode that propagates \emph{without electromagnetic restoring force} --- a massless, charge-neutral collective oscillation.  This has been called the ``acoustic plasmon'' or, in the recent experimental literature (2023, terbium-based metals), the \emph{Demon particle}: a mode that is charge-neutral, acoustic at $q=0$, and invisible to optical probes.\cite{kogar2023demon}
  \item \textbf{Topological surface states.}  In topological insulators, bulk states cancel (Devil bulk) while surface states remain, protected by topology.
  \item \textbf{Dark states in quantum optics.}  A coherent superposition of atomic dipoles designed to have zero net coupling to the radiation field is a ``dark'' or Devil state --- it stores coherence invisibly.
\end{itemize}

In EMTS, a Devil has $Z_{\text{coll}} \approx 0$: the collective state is invisible on the real axis and carries negligible imaginary component too.  Energy is locked into phase-space correlations rather than carried by the field amplitude.

\section{Dynamics: transitions between classes}

The four classes are not static.  An electron system can transit between them driven by temperature, field, or interaction:

\[
  \text{Archangel} \xrightarrow{\text{heat / disorder}} \text{Angel} \xrightarrow{\text{drive / pump}} \text{Demon} \xrightarrow{\text{fine-tune}} \text{Devil}.
\]

In EMTS language each transition corresponds to a \emph{rotation} of the cluster of $z_k$ points:
\begin{itemize}
  \item \textbf{Heating:} thermal fluctuations spread the $\theta_k$ uniformly, reducing $|Z_{\text{coll}}|$ toward zero (Archangel $\to$ Devil).
  \item \textbf{Driving with a laser:} a pump pulse can align all phases, pushing from Demon to Angel (Demon $\to$ Angel, cf.\ ultrafast demagnetization and light-induced superconductivity).
  \item \textbf{Band engineering:} designing a material so that two electron bands have exactly opposite Drude weights creates a stable Devil state; the EMTS sum stays near zero even at low temperature.\cite{kogar2023demon}
\end{itemize}

\section{Summary: the celestial hierarchy in EMTS}

The electron's wave nature means that a collection of electrons is \emph{not} merely a sum of identical particles: phases matter, and their interplay creates qualitatively new states.  The EMTS complex-plane language makes this concrete:
\begin{itemize}
  \item The \textbf{modulus} $|Z_{\text{coll}}|$ measures how ``real'' (observable, matter-like) the collective state is.
  \item The \textbf{argument} $\arg(Z_{\text{coll}})$ measures whether the state is dominantly mass-like ($\arg \approx 0$) or energy-like ($\arg \approx \pi/2$).
  \item \textbf{Angels and Archangels} live near the positive real axis --- large, stable, easily observed.
  \item \textbf{Demons and Devils} live near the origin or the imaginary axis --- fragile, short-lived, detectable only indirectly (energy-loss spectroscopy, optical conductivity, neutron scattering).
\end{itemize}

This taxonomy is not merely descriptive: it provides a unified language for collective electron physics inside the EMTS framework, connecting condensed-matter quasi-particles (plasmons, Cooper pairs, acoustic Demons) to the same complex-plane geometry that organizes fundamental forces, quantum entanglement, and particle trajectories in earlier chapters.

\begin{table}[ht]
\centering
\caption{EMTS celestial hierarchy of collective electron states}
\label{tab:celestial}
\renewcommand{\arraystretch}{1.4}
\begin{tabular}{p{2.2cm} p{3.2cm} p{2.5cm} p{5cm}}
\hline
\textbf{Class} & \textbf{EMTS $Z_{\text{coll}}$} & \textbf{Nature} & \textbf{Examples} \\
\hline
Archangel & $\approx Nr_e\,e^{i\bar\theta}$, tight cluster & Extreme coherence & BEC, superfluid, laser \\
Angel     & Large $|Z|$, moderate spread & Constructive (natural) & Cooper pairs, filled bands \\
Demon     & Small $|Z|$, near imaginary axis & Partial destructive & Bulk plasmon, SPP \\
Devil     & $|Z|\approx 0$, near origin & Extreme destructive & Acoustic plasmon, dark state, topological void \\
\hline
\end{tabular}
\end{table}
