% Part III — Implementation
\chapter{Electron Angels and Demons}
\label{ch:electron-demons}

\section{Motivation: the dual nature of the electron}

The electron is simultaneously a particle and a wave.  As a particle it carries a well-defined mass $m_e$ and charge $-e$; as a wave it is described by a complex-valued wavefunction $\psi(\mathbf{x},t)=|\psi|\,e^{i\phi}$ that obeys the Schr\"odinger (or Dirac) equation.  When many-electron wavefunctions overlap, the phases $\phi_k$ can align or oppose, producing entirely new macroscopic states that bear little resemblance to the bare electron.

Within the EMTS framework the electron sits at
\[
  z_e = m_e + iE_e = r_e\,e^{i\theta_e},
\]
where $r_e = \sqrt{m_e^2+E_e^2}$ is the total mass-energy magnitude and $\theta_e = \arctan(E_e/m_e)$ records the partition between rest mass and kinetic/potential energy.  For a free electron at rest $\theta_e \approx 0$ (real axis), while a high-energy electron is rotated toward the imaginary axis.

A \emph{collective} state of $N$ electrons is then described by the vector sum
\[
  Z_{\text{coll}} = \sum_{k=1}^{N} z_k = \sum_{k=1}^{N} r_k\,e^{i\theta_k}.
\]
The character of this sum --- constructive or destructive --- determines whether the system behaves as an \textbf{Angel} or a \textbf{Demon} in the taxonomy developed below.

\section{Constructive interference: Angels}
\label{sec:angels}

\subsection{Definition}

Consider $N$ electrons whose EMTS phases are \emph{aligned}:
\[
  \theta_k \approx \bar\theta \quad \forall\, k.
\]
The collective vector is then approximately
\[
  Z_{\text{Angel}} \approx \left(\sum_{k=1}^{N} r_k\right) e^{i\bar\theta},
\]
which is a single EMTS point with modulus $\sim Nr_e$ and essentially the same argument as the individual electrons.  The real (mass-like) and imaginary (energy-like) components both scale up together.

\subsection{Physical content}

Phase alignment means that maxima of the individual wavefunctions coincide.  Some familiar realizations:
\begin{itemize}
  \item \textbf{Cooper pairs and superconductivity.}  Below $T_c$, electrons near the Fermi level pair into Cooper pairs with equal and opposite momenta and opposite spins; their pair wavefunction is a single coherent superposition.  In EMTS, the pair amplitude has its phases locked so that $Z_{\text{pair}}$ sits solidly on the real axis (dominant mass, minimal energy dispersion), explaining the gap and lossless transport.\cite{ashcroft_mermin}
  \item \textbf{Coherent many-body ground states.}  In a crystal, Bloch electrons form bands; near a filled lower band the coherent sum of occupied states gives a stable, low-energy configuration --- the background ``sea'' of condensed matter.
  \item \textbf{Stimulated emission.}  In a laser medium, photon-driven transitions reinforce phase alignment of atomic dipoles; the emitted photons are the EM analogue of the Angel state.
\end{itemize}

\subsection{EMTS geometry}

In the complex plane, an Angel state corresponds to a tight \emph{cluster} of $N$ points near a single $z_e$, their vector sum pointing radially outward with length $\sim Nr_e$.  The projection $\Re(Z_{\text{Angel}}) = Nr_e\cos\bar\theta$ is large: Angels are \emph{observable} in the real (mass/matter) sector.

\section{Destructive interference: Demons}
\label{sec:demons}

\subsection{Definition}

Now suppose the phases of the participating electrons are \emph{anti-aligned}:
\[
  \theta_k = \bar\theta + \delta_k, \quad \sum_{k=1}^{N} e^{i\delta_k} \approx 0.
\]
In the simplest two-electron toy model with $\theta_1 = \theta$ and $\theta_2 = \theta + \pi$:
\[
  Z_{\text{Demon}} = r_1 e^{i\theta} + r_2 e^{i(\theta+\pi)} = (r_1 - r_2)\,e^{i\theta}.
\]
When $r_1 = r_2$ the individual mass-energy contributions \emph{cancel exactly}: $Z_{\text{Demon}} = 0$.  Even when cancellation is only partial, the dominant real-axis component is suppressed, and what survives is primarily imaginary --- \emph{energy without mass}.

\subsection{Electrons as plasmons}

The most experimentally dramatic Demon is the \emph{plasmon} --- the collective charge-density oscillation of the electron gas.\cite{ashcroft_mermin}  A classical derivation illustrates the Demon mechanism:

\begin{enumerate}
  \item Each conduction electron in a metal has individual EMTS vector $z_k = m_e + iE_k$.
  \item A charge displacement excites a density wave $n(\mathbf{x},t)=n_0 + \delta n\,\cos(\mathbf{q}\cdot\mathbf{x}-\omega t)$.
  \item The restoring Coulomb forces couple all electrons so that individual particle identities dissolve into a collective mode at the plasma frequency
        \[
          \omega_p = \sqrt{\frac{n_0 e^2}{\epsilon_0 m_e}}.
        \]
  \item This new quasi-particle --- the \emph{plasmon} --- has an effective mass determined by $\omega_p$ and a dispersion $\omega(\mathbf{q})$ very different from the bare electron.
\end{enumerate}

In EMTS, the plasmon is described not by any single $z_k$ but by the \emph{residual} of the collective sum after destructive cancellation.  Its EMTS vector sits near the imaginary axis (large $\theta$, small real component), reflecting the fact that the plasmon has very little ``matter-like'' character --- it is primarily an \emph{energy excitation} propagating through the electron sea.

\subsection{The Demon analogy}

The metaphor of the \emph{Demon} is apt:
\begin{itemize}
  \item Like a demon, the plasmon exploits the absence of order (destructive phase cancellation) to manifest as something qualitatively different from the individual electrons.
  \item It ``hides'' behind the anti-phase interference --- invisible to direct mass measurement, yet capable of transporting energy at $\omega_p$ and mediating optical phenomena at UV frequencies.
  \item Demons are \emph{unnatural} in the sense that they require a specific cancellation to persist; they depend on the maintenance of phase opposition across the participating electrons.
\end{itemize}

\subsection{EMTS geometry}

In the complex plane, the individual electron points spread symmetrically around the origin so that their vector sum $Z_{\text{Demon}}$ is close to (or exactly at) zero.  The plasmon is then the \emph{phase-space excitation} of this near-cancellation --- a small residual that rotates rapidly near the imaginary axis, with $|\Re(Z)| \ll |\Im(Z)|$.

\section{An EMTS taxonomy of collective electron modes}
\label{sec:taxonomy}

The constructive/destructive axis is not binary.  We can organize all collective electron states on a spectrum according to the degree of phase coherence in the EMTS sum.  Four named classes arise naturally, analogous to a celestial hierarchy:

\begin{center}
\renewcommand{\arraystretch}{1.4}
\begin{tabular}{llll}
\hline
\textbf{Name} & \textbf{Phase relation} & \textbf{$|Z_{\text{coll}}|/Nr_e$} & \textbf{Physical archetype} \\
\hline
Archangel & Extreme constructive, $\delta_k\to 0$ & $\to 1$ & BEC, superfluid, lasing mode \\
Angel     & Constructive, partial alignment & $0.5 \text{ to } 1$ & Cooper pairs, band ground state \\
Mixed     & Transitional, partial coherence & $0.1 \text{ to } 0.5$ & Normal metal, thermal electron gas \\
Demon     & Destructive, partial cancellation & $\lesssim 0.1$ & Plasmons, acoustic modes \\
Devil     & Extreme destructive, near-total cancel & $\to 0$ & Acoustic plasmon, demon quasi-particle \\
\hline
\end{tabular}
\end{center}

\subsection{Angels: the natural order}

Constructive interference is the \emph{ground-state tendency}: a system of electrons minimizes its energy by occupying coherently the lowest available modes, with wavefunctions adding positively.  Angels represent equilibrium.  Left to evolve without perturbation, an electron ensemble relaxes toward an Angelic configuration because such states minimize free energy.

\subsection{Archangels: extreme coherence}

When phase alignment is \emph{forced} across a macroscopic number of particles --- as in a Bose-Einstein condensate (BEC), a superfluid, or an optical laser --- the collective EMTS vector points almost perfectly along a single direction in the complex plane, with $|Z|\approx Nr_e$.  This is the Archangel limit:
\[
  Z_{\text{Arch}} = N r_e \, e^{i\bar\theta}, \qquad \text{all } \theta_k \equiv \bar\theta.
\]
Archangels are extraordinary: they require macroscopic phase locking, typically enforced by a symmetry-breaking mechanism (BCS gap equation, Gross-Pitaevskii condensation, stimulated emission).  Their EMTS projection on the real axis is the largest possible, making them the \emph{most observable} collective electron states.

\subsection{Demons: the plasmon family}

Standard plasmons (bulk, surface, acoustic) form the core of the Demon class.  Their salient EMTS property is:
\begin{itemize}
  \item Small $|\Re(Z_{\text{coll}})|$ --- minimal ``matter'' character.
  \item Large $|\Im(Z_{\text{coll}})|$ / fast $\theta$ rotation --- dominant energy character.
  \item Finite lifetime: phase opposition is maintained only for the coherence time $\tau_p$; scattering and damping (Landau damping) restore the Angel ground state.
\end{itemize}

The surface plasmon polariton (SPP) is a Demon that lives at the interface between a conductor and a dielectric, coupling to photons --- a mixed photonic/electronic Demon with EMTS vector straddling both electromagnetic and material quadrants.

\subsection{Devils: extreme destructive states}

The Devil limit is total cancellation $Z \to 0$.  This requires fine-tuning or protective symmetry.  Known physical candidates:

\begin{itemize}
  \item \textbf{Acoustic plasmon \& the ``electronic Demon.''} In a multi-band metallic system where two bands with nearly equal but opposite effective-mass contributions coexist, destructive cancellation of the conventional Drude weight can yield a mode that propagates \emph{without electromagnetic restoring force} --- a massless, charge-neutral collective oscillation.  David Pines predicted exactly this mode in 1956\cite{pines1956}: he showed that in a two-component electron fluid the conventional plasmon (a Devil in the EMTS taxonomy) acquires an acoustic branch that carries no net charge and couples minimally to photons --- a ``demon'' in the literal sense of an entity that hides from direct observation.  Nearly seven decades later the mode was observed experimentally in terbium-based intermetallic compounds,\cite{kogar2023demon} confirming Pines' original prediction and its charge-neutral, acoustic character at $q=0$.
  \item \textbf{Topological surface states.}  In topological insulators, bulk states cancel (Devil bulk) while surface states remain, protected by topology.
  \item \textbf{Dark states in quantum optics.}  A coherent superposition of atomic dipoles designed to have zero net coupling to the radiation field is a ``dark'' or Devil state --- it stores coherence invisibly.
\end{itemize}

In EMTS, a Devil has $Z_{\text{coll}} \approx 0$: the collective state is invisible on the real axis and carries negligible imaginary component too.  Energy is locked into phase-space correlations rather than carried by the field amplitude.

\section{Dynamics: transitions between classes}

The four classes are not static.  An electron system can transit between them driven by temperature, field, or interaction:

\[
  \text{Archangel} \xrightarrow{\text{heat / disorder}} \text{Angel} \xrightarrow{\text{drive / pump}} \text{Demon} \xrightarrow{\text{fine-tune}} \text{Devil}.
\]

In EMTS language each transition corresponds to a \emph{rotation} of the cluster of $z_k$ points:
\begin{itemize}
  \item \textbf{Heating:} thermal fluctuations spread the $\theta_k$ uniformly, reducing $|Z_{\text{coll}}|$ toward zero (Archangel $\to$ Devil).
  \item \textbf{Driving with a laser:} a pump pulse can align all phases, pushing from Demon to Angel (Demon $\to$ Angel, cf.\ ultrafast demagnetization and light-induced superconductivity).
  \item \textbf{Band engineering:} designing a material so that two electron bands have exactly opposite Drude weights creates a stable Devil state; the EMTS sum stays near zero even at low temperature.\cite{kogar2023demon}
\end{itemize}

\section{A Pindaric Flight: Protons, Color, and the Strong Force}
\label{sec:proton-demons}

\emph{This section is deliberately speculative.  It extends the Angels-and-Demons framework beyond its established condensed-matter ground and asks whether the same wave-interference logic governs the strong nuclear force.  The reader is invited to treat what follows as a conjecture rather than a derivation.}

\subsection{The proton as a collective state}

The proton is not an elementary particle.  It is made of three quarks ($uud$), each carrying one of three \emph{color charges} --- called Red, Green, and Blue --- bound together by the exchange of gluons.  What emerges is electrically charged, massive, and confined: no free quark has ever been observed.  The question we raise here is: can the same phase-interference mechanism that produces plasmons from electrons produce the \emph{strong force} from quarks?

In EMTS, each quark carries a mass-energy vector $z_q = m_q + i E_q$ as before.  But it also carries an \emph{internal color phase} $\phi_c$, which we treat as an additional angular degree of freedom orthogonal to the EMTS time-phase $\theta = \omega t$ and distinct from the single-electron wavefunction phase $\phi$ introduced in Section~\ref{sec:demons}.  Three angles thus appear in this chapter and the reader should keep them separate: $\phi$ (microscopic wavefunction phase of a single electron), $\theta$ (the EMTS mass--energy angle, identified with time via $\theta=\omega t$, and used throughout this book), and $\phi_c$ (internal color phase living in the $SU(3)$ color space, introduced only in this section).  We write the full quark state schematically as
\[
  \tilde{z}_q = z_q \cdot e^{i\phi_c},
\]
where $\phi_c \in \{0,\, 2\pi/3,\, 4\pi/3\}$ for the Red, Green, and Blue quarks respectively.

\subsection{The color-Devil condition}

Consider the three quarks in a proton and form their color-phase sum:
\[
  Z_{\text{color}} = e^{i\cdot 0} + e^{i\cdot 2\pi/3} + e^{i\cdot 4\pi/3}
  = 1 + \left(-\tfrac{1}{2}+i\tfrac{\sqrt{3}}{2}\right) + \left(-\tfrac{1}{2}-i\tfrac{\sqrt{3}}{2}\right) = 0.
\]
This is the \textbf{Devil condition} applied to color space.  The three quark color phases are placed exactly $120^\circ$ apart on the unit circle and cancel completely --- a perfect three-way destructive interference.  In EMTS terms:
\begin{itemize}
  \item $|Z_{\text{color}}| = 0$: the proton has no net color charge.  It is ``color-invisible.''
  \item The individual quarks are not observable ($|z_q| \neq 0$, color charge exposed); only the Devil combination $Z_{\text{color}} = 0$ can propagate freely.
\end{itemize}
This is \emph{color confinement} restated in the language of the Angel/Demon taxonomy: \textbf{Nature requires every free particle to satisfy the color-Devil condition.}  What we call ``the strong force'' is, in this reading, the mechanism that enforces it.\cite{gellmann1964}

\subsection{Why this is deeper than the electron Demon}

In the electron case the Demon (plasmon) arises as a collective excitation \emph{above} an Angel ground state.  Destructive interference is an excited, transient phenomenon; the ground state is constructive.  For quarks the situation is inverted:
\begin{itemize}
  \item The color-Devil condition is the \emph{ground state requirement} --- particles that do not satisfy it are confined and never free.
  \item The ``color Angel'' limit --- a state with net color charge --- would correspond to a free quark.  It has never been observed at low energies, though it is approached in the quark-gluon plasma (QGP) at extreme temperature and density (see below).
\end{itemize}
The electron Demons are fragile and short-lived; the color Devil is permanent at low energy.  The strong force is \emph{so} strong precisely because the Devil condition is enforced with energy cost that grows with separation (confinement), unlike the Coulomb force, which weakens with distance.

\subsection{Gluons as color Demons}

In the electron gas, the Demon (plasmon) is a collective excitation of the electron field --- a Demon that mediates energy transfer without carrying net charge.  The gluon plays an analogous role for the color field, but with a crucial difference:
\begin{itemize}
  \item A \textbf{photon} is charge-neutral --- the EM Demon does not carry the charge it mediates.
  \item A \textbf{gluon} carries \emph{color charge} itself (e.g., a red-anti-green gluon).  The color Demon is \emph{self-interacting}.
\end{itemize}
This self-interaction of the color Demon is the origin of \emph{asymptotic freedom} (at short distances, gluon self-coupling weakens) and \emph{confinement} (at long distances, gluon self-coupling creates a flux tube whose energy grows linearly with separation).  In EMTS, the self-interacting Demon feeds back into the color-phase sum: every gluon emission re-colors one quark, immediately reshuffling the color phases to maintain $Z_{\text{color}} = 0$.

\subsection{Pions as two-body color Devils and the nuclear force}

A pion ($\pi^+= u\bar{d}$, $\pi^- = \bar{u}d$, $\pi^0 = u\bar{u}/d\bar{d}$) is a quark-antiquark pair.  Its color phase sum is simpler:
\[
  Z_{\text{color}}^{\pi} = e^{i\phi_c} + e^{i(\phi_c + \pi)} = 0,
\]
since a quark of color $\phi_c$ and its antiquark of anti-color $\phi_c + \pi$ destructively cancel in a \emph{two-body} color Devil.  The pion is much lighter than the proton (mass $\approx 140$ MeV vs.\ $938$ MeV) because it is the pseudo-Goldstone boson of spontaneously broken chiral symmetry: it costs very little energy to create a two-body color Devil, making it the natural \emph{carrier} of the residual nuclear force between protons and neutrons.  In EMTS, the pion is a color-Devil mediator: its $Z_{\text{color}} = 0$ lets it travel between hadrons without violating confinement, transferring momentum and binding the nucleus.

\subsection{The quark-gluon plasma: melting the color Devil}

At temperatures above $T_c \approx 150$ MeV (achieved in heavy-ion collisions at RHIC and LHC), hadronic matter deconfines into a \emph{quark-gluon plasma} (QGP).  In EMTS language:
\begin{itemize}
  \item Below $T_c$: quarks are locked into color-Devil configurations (hadrons).  The color-phase sum is $Z_{\text{color}} = 0$ for every observable state.
  \item Above $T_c$: thermal fluctuations are energetic enough to break the color-Devil constraint.  Color-charged quarks roam freely --- the system acquires a nonzero, fluctuating $Z_{\text{color}}$ in each local region.  This is the color-Angel (or color-Archangel) limit.
\end{itemize}
The QGP phase transition is therefore the color analogue of breaking Cooper pairs above $T_c$ in a superconductor: an Archangel (BCS superfluid) melts into a normal electron gas (Angel/Demon) as temperature rises.  Here the color-Devil (hadron) melts into a color-Angel (free quark) as temperature rises.

\subsection{A tentative EMTS conjecture}

Bringing the thread together: the fundamental forces can be tentatively classified by \emph{which type of EMTS interference condition they enforce}:

\begin{center}
\renewcommand{\arraystretch}{1.4}
\begin{tabular}{lll}
\hline
\textbf{Force} & \textbf{Enforced condition} & \textbf{Free-particle rule} \\
\hline
Electromagnetism & No condition on phase & Charged particles propagate freely \\
Weak force       & $SU(2)$ phase alignment (Angel) & Broken at low $E$ by Higgs \\
Strong force     & Color-Devil ($Z_{\text{color}}=0$) & Only color-neutral states are free \\
Gravity          & Spacetime curvature (EMTS radial) & All massive objects curve $r$ \\
\hline
\end{tabular}
\end{center}

In this reading, the strong force is \emph{the Devil force}: it is the force that imposes the most extreme destructive-interference condition on its constituents, making its bound states ($Z_{\text{color}}=0$) uniquely stable and its individual constituents (quarks) permanently hidden.

Whether this is merely a suggestive analogy or points toward a deeper unification within the EMTS framework remains an open question --- one we leave to future chapters and future physicists.

\section{Summary: the celestial hierarchy in EMTS}

The electron's wave nature means that a collection of electrons is \emph{not} merely a sum of identical particles: phases matter, and their interplay creates qualitatively new states.  The EMTS complex-plane language makes this concrete:
\begin{itemize}
  \item The \textbf{modulus} $|Z_{\text{coll}}|$ measures how ``real'' (observable, matter-like) the collective state is.
  \item The \textbf{argument} $\arg(Z_{\text{coll}})$ measures whether the state is dominantly mass-like ($\arg \approx 0$) or energy-like ($\arg \approx \pi/2$).
  \item \textbf{Angels and Archangels} live near the positive real axis --- large, stable, easily observed.
  \item \textbf{Demons and Devils} live near the origin or the imaginary axis --- fragile, short-lived, detectable only indirectly (energy-loss spectroscopy, optical conductivity, neutron scattering).
\end{itemize}

This taxonomy is not merely descriptive: it provides a unified language for collective electron physics inside the EMTS framework, connecting condensed-matter quasi-particles (plasmons, Cooper pairs, acoustic Demons) to the same complex-plane geometry that organizes fundamental forces, quantum entanglement, and particle trajectories in earlier chapters.

\begin{table}[ht]
\centering
\caption{EMTS celestial hierarchy of collective electron states}
\label{tab:celestial}
\renewcommand{\arraystretch}{1.4}
\begin{tabular}{p{2.2cm} p{3.2cm} p{2.5cm} p{5cm}}
\hline
\textbf{Class} & \textbf{EMTS $Z_{\text{coll}}$} & \textbf{Nature} & \textbf{Examples} \\
\hline
Archangel & $\approx Nr_e\,e^{i\bar\theta}$, tight cluster & Extreme coherence & BEC, superfluid, laser \\
Angel     & Large $|Z|$, moderate spread & Constructive (natural) & Cooper pairs, filled bands \\
Demon     & Small $|Z|$, near imaginary axis & Partial destructive & Bulk plasmon, SPP \\
Devil     & $|Z|\approx 0$, near origin & Extreme destructive & Acoustic plasmon, dark state, topological void \\
\hline
\end{tabular}
\end{table}
