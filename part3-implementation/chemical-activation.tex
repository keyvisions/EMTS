% Part III — Implementation
\chapter{Chemical Activation Analogue}
\label{ch:chemical-activation}

This section borrows the logic of chemical activation diagrams and transplants it into the mass–energy–phase plane.

\section{Analogy}
In chemistry, an activation diagram plots potential energy vs. reaction coordinate. Here, the “reaction coordinate” is a path in \((r,\theta)\): initial \(z_i=r_i e^{i\theta_i}\), final \(z_f=r_f e^{i\theta_f}\), and a barrier as a dynamically disfavoured region.

\section{Activation mass–energy}
The activation energy becomes an activation mass–energy: the extra \(|z|\) or angular displacement needed to connect two states.

\section{Manipulating mass/energy}
\begin{itemize}
  \item \textbf{Catalysis analogue:} Introduce an intermediate path bending through a quadrant with a lower barrier; mediators or fields change the allowed trajectory so the peak \(|z|\) is smaller.
  \item \textbf{Phase-assisted transitions:} Because \(\theta\) is cyclic, one can wrap around instead of going straight over, akin to tunnelling.
  \item \textbf{Density-tuned activation:} If \(\omega(\rho,r)\) changes effective heights, altering local density can lower the barrier via phase-resonance.
\end{itemize}

\section{Diagrammatic representation}
Plot total \(|z|=r\) vs. path length along \((r,\theta)\); different complex-plane paths yield different activation curves.

\section{Formalization}
Define an activation functional for a path \(\gamma\):
\[
\mathcal{A}[\gamma] = \max_{s\in\gamma} \bigl[ r(s) - \min(r_i,r_f) \bigr],
\]
and search for paths minimizing \(\mathcal{A}\) subject to complex-plane conservation laws. Catalysts, fields, or density changes are deformations of \(U(r,\theta)\) that reduce \(\mathcal{A}\).
