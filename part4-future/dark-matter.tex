% Part IV — Future
\chapter{Dark Matter in the Complex Plane}

Building on the EMTS framework (Chapter~\ref{ch:emts-framework}) and quadrant structure (Chapter~\ref{ch:fundamental-forces}), we explore how dark matter naturally fits into the complex plane formalism through six complementary mechanisms. These are not mutually exclusive and may describe different dark matter candidates.

\section{Off-axis or virtual trajectories}

Dark matter might correspond to \textbf{trajectories that never cross the real axis}, analogous to virtual particles in quantum field theory. These would be states with
\[
z_{\text{DM}} = m_{\text{DM}} + iE_{\text{DM}},
\]
where the real projection \(m_z = \projReal\) remains small or effectively zero at observable times \(\theta\), making them gravitationally present (via \(r\)) but electromagnetically invisible.

Such states contribute to the stress-energy tensor \(\langle T_{\mu\nu}\rangle\) through their magnitude \(r = |z|\), affecting spacetime curvature, while their phase \(\theta\) keeps them perpetually out of phase with electromagnetic interactions.

\section{Phase-locked or resonant modes}

Drawing from the resonance and entanglement framework, dark matter could occupy \textbf{phase-locked states} with \(\Delta\theta\) that never aligns with the electromagnetic quadrant (\quadEM). Such states would:
\begin{itemize}
\item Contribute to gravitational potential via their radius \(r = |z|\)
\item Remain decoupled from electromagnetic interactions because their phase windows \(W_f(\theta)\) have negligible overlap with the EM sector
\end{itemize}

The condition for invisibility is
\[
\int_{\theta_{\text{EM}}} P_{\text{EM}}(\theta)\,|\Psi_{\text{DM}}(\theta)|^2\,d\theta \approx 0,
\]
where \(P_{\text{EM}}(\theta)\) is the electromagnetic sector projector.

\section{High-radius, low-coupling orbits}

From the Standard Model mapping, particles are characterized by \((r, \omega)\) pairs. Dark matter could be described by
\[
z_{\text{DM}} = r_{\text{large}} e^{i\omega_{\text{slow}}t},
\]
with large spatial scale \(r\) (corresponding to spread-out density) and slow angular frequency \(\omega\) (corresponding to low interaction rate), placing it in a regime where:
\begin{itemize}
\item Gravitational coupling \(\propto r\) remains strong
\item Electromagnetic coupling \(g_{\text{EM}}(\theta,r)\) vanishes due to phase mismatch or radius-dependent suppression
\end{itemize}

This mechanism is related to the gauge coupling functions
\[
g_s(\theta, r) = g_s^{(0)} \cdot \mathcal{F}_s(\theta) \cdot \mathcal{G}_s(r),
\]
where \(\mathcal{F}_s(\theta)\) provides phase selectivity and \(\mathcal{G}_s(r)\) introduces scale dependence. Dark matter corresponds to the regime where \(\mathcal{F}_{\text{EM}}(\theta_{\text{DM}}) \ll 1\) while \(\mathcal{G}_{\text{grav}}(r_{\text{DM}})\) remains appreciable.

\section{Quadrant isolation}

Using the quadrant structure from Chapter~\ref{ch:fundamental-forces}, dark matter might:
\begin{itemize}
\item Occupy \textbf{\quadGrav\ (Gravitational)} exclusively, never entering \quadEM\ (Electromagnetic)
\item Reside in a \textbf{forbidden transition zone} between quadrants, where paths cannot close into observable particles but still contribute to \(\langle T_{\mu\nu}\rangle\) in the Einstein field equations
\end{itemize}

The potential \(U_{\text{quad}}(\theta)\) creates barriers between sectors:
\[
U_{\text{quad}}(\theta) = \sum_{n=1}^{4} V_n \left[1 - \cos\left(4\theta - \frac{\pi(n-1)}{2}\right)\right],
\]
with minima at quadrant centers. Dark matter states could be:
\begin{enumerate}
\item Trapped in Quadrant IV with insufficient energy to reach Quadrant I
\item Localized at barrier maxima between quadrants (analogous to domain walls)
\end{enumerate}

\section{Density-driven phase speed modification}

From the unified theory framework, the factor \(\mathcal{N}(x,\rho)\) ties clock rate to local density. Dark matter could be a \textbf{self-consistent solution} where
\[
\omega(\rho_{\text{DM}}, r) \ll \omega_{\text{visible}},
\]
so its phase evolution is too slow to synchronize with baryonic matter, keeping it perpetually out of phase with electromagnetic detection windows but gravitationally active via backreaction on \(G_{\mu\nu}\).

The modified evolution equation becomes
\[
i\hbar\,\mathcal{N}(x,\rho_{\text{DM}})\,\partial_t \Psi_{\text{DM}} = \hat{H}_{\text{DM}}\Psi_{\text{DM}},
\]
where \(\mathcal{N}(x,\rho_{\text{DM}}) \gg 1\) effectively slows down the local clock, creating a temporal decoherence from the visible sector.

\section{Missing geometric states}

Inspired by the periodic table structure, dark matter could represent a \textbf{gap in the allowed \((r,\theta)\) spectrum}: a state required by closure or symmetry rules but never appearing in visible channels because:
\begin{itemize}
\item Its winding number \(n\) in \(e^{in\theta}\) is exotic (e.g., half-integer or irrational in some extended framework)
\item Its potential \(U_{\text{quad}}(\theta)\) traps it in a sector invisible to electromagnetic probes
\end{itemize}

The eigenmodes of the phase Hamiltonian
\[
\hat{H}_\theta = -\frac{\hbar^2}{2I_\theta}\,\partial_\theta^2 + U_{\text{quad}}(\theta)
\]
may include states with quantum numbers that forbid transitions to electromagnetic-active states, yet these states still carry mass-energy and thus gravitate.

\section{Composite interpretation}

A realistic dark matter sector may involve \textbf{multiple mechanisms}:
\begin{center}
\begin{tabular}{lll}
\hline
\textbf{Interpretation} & \textbf{Key Property} & \textbf{Why Invisible} \\
\hline
Virtual/off-axis & \(m_z \approx 0\) at observable \(\theta\) & Never projects to real axis \\
Phase-locked & \(\Delta\theta\) out of EM window & No EM quadrant overlap \\
High-\(r\), low-\(\omega\) & Large scale, slow cycle & Coupling \(g(\theta,r)\) suppressed \\
\quadGrav\ only & Gravitational sector confined & Never enters \quadEM \\
Density-modified clock & \(\omega(\rho_{\text{DM}})\) too slow & Phase decoherence from visible \\
Missing geometric state & Gap in \((r,\theta)\) spectrum & Forbidden/weak transition \\
\hline
\end{tabular}
\end{center}

\section{Testable predictions}

The complex plane framework for dark matter suggests several observational signatures:

\subsection{Phase-independent gravitational lensing}
If dark matter is phase-decoupled, then gravitational lensing (purely \(r\)-dependent) should show mass distributions that do \textbf{not} correlate with any electromagnetic phase windows. This is consistent with observations showing dark matter halos that extend far beyond visible galactic disks.

\subsection{Missing resonances in direct detection}
Direct detection experiments search for dark matter-nucleon scattering. In the phase-locked scenario, dark matter at \(\theta_{\text{DM}} \in \text{Quadrant IV}\) would have suppressed overlap with nuclear matter in Quadrants I--II, explaining null results:
\[
\sigma_{\text{DM-nucleon}} \propto \left|\int d\theta\, \Psi_{\text{DM}}^*(\theta)\,P_{\text{EM}}(\theta)\,\Psi_{\text{nucleon}}(\theta)\right|^2 \ll \sigma_{\text{expected}}.
\]

\subsection{Anomalous gravitational signatures}
Regions with high \(\rho_{\text{DM}}\) may exhibit modified \(\mathcal{N}(x,\rho)\), leading to:
\begin{itemize}
\item Local time dilation effects in dark matter-dominated regions
\item Modified dispersion relations for photons traversing dark matter halos
\item Possible phase coherence effects in colliding dark matter structures
\end{itemize}

\subsection{Indirect searches via phase transitions}
Extreme conditions (early universe, black hole mergers, neutron star collisions) might temporarily drive dark matter states into electromagnetic quadrants, creating transient signals. The transition probability scales as
\[
P_{\text{transition}} \sim \exp\left(-\frac{\Delta U_{\text{quad}}}{k_B T_{\text{eff}}}\right),
\]
where \(\Delta U_{\text{quad}}\) is the barrier height between gravitational and electromagnetic quadrants.

\section{Open questions}

This geometric framework for dark matter raises several theoretical questions:
\begin{enumerate}
\item What determines the distribution of \(r\) and \(\theta\) for primordial dark matter?
\item Can phase-locked dark matter cluster gravitationally while maintaining phase coherence?
\item Do dark matter self-interactions arise from \(\theta\)-overlap between different dark matter species?
\item What role does dark matter play in the early universe phase structure when all quadrants may be thermally accessible?
\item Could dark energy correspond to a uniform background phase \(\theta_{\Lambda}\) distinct from both visible and dark matter?
\end{enumerate}

\section{Connection to cosmology}

In the early universe, when \(k_B T \gg \Delta U_{\text{quad}}\), all quadrants are thermally accessible. As the universe cools, a phase transition occurs where states "freeze out" into their respective quadrants:
\[
T < T_{\text{freeze}} \sim \frac{\Delta U_{\text{quad}}}{k_B}.
\]

Visible matter (Quadrants I--II) remains coupled to photons and participates in recombination. Dark matter (Quadrant IV) decouples earlier, matching the observed dark matter relic abundance:
\[
\Omega_{\text{DM}} \sim \frac{\int_{\text{Quad IV}} d\theta\,\rho(\theta)}{\int_{\text{all quad}} d\theta\,\rho(\theta)} \approx 0.85,
\]
consistent with cosmological observations.

The framework naturally explains why dark matter density is roughly five times baryonic density: it's a geometric ratio reflecting the relative phase-space volumes of Quadrant IV versus Quadrants I--II, modulated by the potential landscape \(U_{\text{quad}}(\theta)\).
