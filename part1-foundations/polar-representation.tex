\chapter{Polar Representation and Its Implications}
\label{ch:polar-representation}

Building on the core framework (Chapter~\ref{ch:emts-framework}), we explore deeper implications of the polar representation $\EMTSz$. This chapter examines how the modulus $r$ and argument $\theta$ relate to physical measurements and quantum phenomena.

\section{Extended physical interpretation}

The modulus $r = \sqrt{m^2+E^2}$ and argument $\theta = \arctan(E/m)$ were defined and given their primary physical interpretation in Section~\ref{sec:polar-rep} of Chapter~\ref{ch:emts-framework}. In SI units ($c \neq 1$) these generalise to $r = \sqrt{(mc^2)^2+E^2}$ and $\theta = \arctan(E/mc^2)$. This chapter builds directly on that foundation, extending the physical interpretation rather than re-deriving it.

\paragraph{Space and Time.}
The key relation $\phaseTime$ with $E=\hbar\omega$ means probabilities or densities linked to projections oscillate with $\omega$. The projection $m_z=\Re(z)=\projReal$ can repeat with period $2\pi$ even as the underlying phase history differs.

\paragraph{Projection as Measurement.}
With the unit vector $\lvert \psi \rangle = \cos\theta\,\lvert m \rangle + i\sin\theta\,\lvert E \rangle$, a mass measurement uses $P_m$ and yields $\mathbb{P}(m)=\cos^2\theta$, mirroring $m_z=\projReal$. Standard treatments of such two-level mappings can be found in Griffiths\cite{griffiths_qm}.
