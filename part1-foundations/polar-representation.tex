\chapter{Polar Representation and Its Implications}
Mapping mass and energy in the complex plane broadens our perspective on physical reality. Consider a point $z$ expressed in polar form:
\[
z = r(\cos\theta + i\sin\theta).
\]
Here $r$ is the modulus and $\theta$ the argument. Assign physical meaning to $r$ and $\theta$:

\paragraph{Modulus $r$ (Unified Magnitude).}
\[
r = \sqrt{(mc^2)^2 + E^2} \quad \text{or} \quad r = \sqrt{m^2 + \left(\tfrac{E}{c^2}\right)^2}.
\]

\paragraph{Argument $\theta$ (Unified Phase).}
\[
\theta = \arctan\!\left(\frac{E}{mc^2}\right).
\]

\paragraph{Space and Time.}
Interpret $\theta=\omega t$ with $E=\hbar\omega$. Then probabilities or densities linked to projections oscillate with $\omega$. The projection $m_z=\Re(z)=r\cos\theta$ can repeat with period $2\pi$ even as the underlying phase history differs.

\paragraph{Projection as Measurement.}
With the unit vector $\lvert \psi \rangle = \cos\theta\,\lvert m \rangle + i\sin\theta\,\lvert E \rangle$, a mass measurement uses $P_m$ and yields $\mathbb{P}(m)=\cos^2\theta$, mirroring $m_z=r\cos\theta$. Standard treatments of such two-level mappings can be found in Griffiths\cite{griffiths_qm}.
