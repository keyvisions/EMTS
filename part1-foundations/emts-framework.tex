% Part I — Foundations
\chapter{The EMTS Framework: Core Definitions}
\label{ch:emts-framework}

Having introduced the complex plane and its historical significance, we now establish the foundational elements of the Energy-Mass-Time-Space (EMTS) framework. This chapter consolidates the core mathematical definitions and physical interpretations that underpin the entire theory.

\section{The complex plane mapping}
\label{sec:complex-mapping}

The EMTS framework posits that physical states can be represented as points in the complex plane:
\[
z = m + iE,
\]
where:
\begin{itemize}
\item $m$ is the mass component (mapped to the real axis)
\item $E$ is the energy component (mapped to the imaginary axis)
\item $i = \sqrt{-1}$ is the imaginary unit
\end{itemize}

This choice reflects the intuition that mass is "more tangible" than energy in everyday experience, though the mapping could be reversed without loss of generality. What matters is the \emph{relationship} between the components, not which axis they occupy.

\section{Polar representation}
\label{sec:polar-rep}

Every complex point $z$ can be expressed in polar form:
\[
z = r e^{i\theta} = r(\cos\theta + i\sin\theta),
\]
where:
\begin{itemize}
\item $r = |z| = \sqrt{m^2 + E^2}$ is the \textbf{modulus} (unified magnitude)
\item $\theta = \arg(z) = \arctan(E/m)$ is the \textbf{argument} (unified phase)
\end{itemize}

\subsection{Physical interpretation of $r$ and $\theta$}

The modulus $r$ represents the total mass-energy content:
\[
r = \sqrt{(mc^2)^2 + E^2} \quad \text{or in natural units} \quad r = \sqrt{m^2 + E^2}.
\]

The argument $\theta$ encodes the \textbf{partition} between mass and energy:
\begin{align}
m &= r\cos\theta, \\
E &= r\sin\theta.
\end{align}

Crucially, we identify $\theta$ with time evolution:
\[
\theta(t) = \omega t,
\]
where $\omega$ is the angular frequency related to energy by $E = \hbar\omega$ (Planck's quantum hypothesis).

\subsection{The projection principle}
\label{sec:projection-principle}

\textbf{Central postulate:} Physical reality corresponds to the \emph{projection} of the complex state onto the real axis:
\[
\text{Observable reality} = \Re(z) = m = r\cos\theta.
\]

This projection is \textbf{many-to-one}: infinitely many complex states
\[
\{z_n = m + iE_n\}_{n}
\]
all project to the same observable value $m$. The imaginary component $E$ represents hidden degrees of freedom that influence dynamics but are not directly observed.

\paragraph{Before collapse: a two-dimensional world.}
Prior to any projection the full state is the complex number $z = re^{i\theta}$, described by exactly
two real parameters --- the spatial scale $r$ and the time-phase $\theta$. The pre-collapse world
is intrinsically \emph{two-dimensional}; no additional spatial coordinates exist at this level.

\paragraph{Collapse creates dimensionality.}
A single projection event $z \to \Re(z) = r\cos\theta$ yields one real number: a point on a line.
But an observer is built from \emph{many} projections of \emph{many} independent states. Each
projection event pins down one real value along one independent degree of freedom. When enough
independent projections accumulate, the observer can \emph{triangulate}: three independent
real-axis readouts suffice to locate an event in everyday experience, which is why space appears
three-dimensional. More precisely:
\begin{itemize}
  \item $D = 1$ independent projection: a point on a line.
  \item $D = 3$ independent projections: a point in space --- the familiar 3D world.
  \item $D = N$ independent projections: an effective $N$-dimensional space, accessible in principle
        to a sufficiently rich measuring apparatus (Chapter~\ref{ch:mapping-reality}).
\end{itemize}
Three-dimensional space is therefore not a pre-existing container. It is the \emph{minimum number of
independent real-axis readouts} required to locate matter, and its value $D=3$ is an empirical fact
about which degrees of freedom decohere at macroscopic energy scales. Crucially, this argument applies
only to \emph{particle-like} observers --- those who interact with the world through localised
projection events. A wave-like observer, one that never forces a collapse, would have no projection
events from which to assemble spatial coordinates and would perceive the world as inherently
two-dimensional (see Section~\ref{sec:wave-universe-dimensionality}).

\section{Fundamental relationships}
\label{sec:fundamental-relations}

\subsection{Time as phase}
\label{sec:time-as-phase}

The identification $\theta = \omega t$ unifies time evolution with rotation in the complex plane. A state evolves as:
\[
z(t) = r e^{i\omega t},
\]
and the observable projection oscillates:
\[
m(t) = r\cos(\omega t),
\]
explaining wave-like behaviour from the geometry of rotation.

The argument $\theta = \omega t$ is the natural time variable of the EMTS framework, but it carries a
dual character that distinguishes it from the ordinary parameter $t$:

\paragraph{Monotonic growth — the arrow of time.}
As $t$ advances, $\theta = \omega t$ increases without bound: $\theta \in (-\infty, +\infty)$.  This
unbounded growth is the EMTS expression of the \emph{arrow of time}.  No two physically distinct moments
have the same $\theta$ on the universal timeline; the total accumulated phase is an ever-increasing record
of elapsed history.  In this sense $\theta$ is a richer time variable than $t$ itself: it carries
both the elapsed duration (via its magnitude) and the current energy scale (via $\omega = E/\hbar$, so
higher-energy states evolve faster).

\paragraph{Cyclic behaviour — the wave nature of matter.}
Although $\theta$ grows without limit, its effect on the physical state is $2\pi$-periodic:
\[
z(\theta + 2\pi) = r e^{i(\theta + 2\pi)} = r e^{i\theta} = z(\theta).
\]
The observable projection $m = r\cos\theta$ recurs with period $2\pi$ in $\theta$, or equivalently with
period $T = 2\pi/\omega$ in $t$.  This periodicity is the geometric origin of \emph{wave-like behaviour}:
a particle is not oscillating because something is vibrating in ordinary space, but because the EMTS phase
angle is cycling through the same projection values repeatedly.  The wave is the shadow of a rotation.

The two aspects together give $\theta$ its full physical meaning:
\begin{itemize}
  \item The \textbf{unwrapped} $\theta \in \mathbb{R}$ encodes absolute time and energy history.
  \item The \textbf{wrapped} $\theta \bmod 2\pi \in [0, 2\pi)$ encodes the instantaneous
        mass-energy partition and the force quadrant (Section~\ref{sec:force-quadrants}).
\end{itemize}
Quantization arises precisely from demanding single-valuedness of the wavefunction under the cyclic
part: $\Psi(r, \theta + 2\pi) = \Psi(r, \theta)$ forces integer winding numbers $n$, giving discrete
energy spectra $E_n = n\hbar\omega$.

\subsection{Energy-frequency connection}

From quantum mechanics:
\[
E = \hbar\omega \quad \Rightarrow \quad \omega = \frac{E}{\hbar}.
\]
Thus the rotation rate in the complex plane is proportional to energy: a more energetic state rotates
faster, ages faster in phase, and has a shorter period $T = h/E$.  The time variable $\theta$ is
therefore not universal — it runs at a rate set by the energy of each individual state.

\subsection{Space as modulus}
\label{sec:space-as-modulus}

While $\theta$ encodes time, the modulus $r$ encodes \emph{space} — specifically, the accessible spatial
scale of the state.  This identification rests on three interlocking arguments.

\paragraph{1. De Broglie wavelength.}
For a particle with momentum $p$, the de Broglie relation gives a spatial wavelength
\[
  \lambda = \frac{h}{p} = \frac{2\pi\hbar}{p}.
\]
In the EMTS complex plane, when kinetic energy is included the total relativistic energy is
$E_{\rm tot} = \sqrt{p^2 c^2 + m^2 c^4}$, so the EMTS modulus becomes
\[
  r = \frac{1}{c}\sqrt{(pc)^2 + (mc^2)^2} = \frac{|p^\mu|}{c},
\]
the magnitude of the 4-momentum divided by $c$.  Since $r \propto p \propto 1/\lambda$, a larger
modulus corresponds to a \emph{shorter wavelength} — finer spatial resolution.  Conversely, a small
$r$ (non-relativistic, nearly-at-rest state) corresponds to a large spatial wavelength: the state is
spread over a large region.  The modulus $r$ is therefore an \emph{inverse spatial scale}: it
measures how well the state is localised in space.

\paragraph{2. Compton wavelength and rest mass.}
At rest ($p = 0$), all of the modulus comes from the rest mass: $r = mc$.  The corresponding spatial
scale is the Compton wavelength
\[
  \lambda_C = \frac{\hbar}{mc} = \frac{\hbar}{r}.
\]
The Compton wavelength is the minimum spatial extent of a particle; below it, quantum field
creation-and-annihilation processes dominate and the single-particle description breaks down.  A particle
with larger $r$ (larger rest mass) has a smaller Compton wavelength and is more spatially localised — it
occupies a smaller region of space.  The modulus thus sets the \emph{characteristic spatial size} of
the state.

\paragraph{3. Curvature and gravitational radius.}
In general relativity, the Schwarzschild radius of a body with mass-energy $r$ (in natural units)
is
\[
  r_s = \frac{2G_N r}{c^4}.
\]
Larger $r$ curves spacetime more strongly, creating a larger gravitational well — a larger region of
space dominated by that body's field.  The modulus therefore couples directly to the geometry of space:
it controls the spatial extent of the gravitational influence.

\paragraph{$r$ before vs.\ after collapse.}
A crucial distinction separates the role of $r$ on each side of a projection event:
\begin{itemize}
  \item \textbf{Before collapse:} $r$ is a single scalar --- the mass-energy magnitude. It sets a
        \emph{characteristic spatial scale} (Compton wavelength, gravitational reach), but it is
        not a coordinate. The pre-collapse state lives in the 2D plane $(r,\theta)$; there is no
        ``$x$-direction'' or ``$y$-direction'' at this level.
  \item \textbf{After collapse:} each projection event $z_k \to r_k\cos\theta_k$ contributes one
        real number to the observer's record. A collection of $D$ independent such records
        \emph{assembles} a $D$-dimensional coordinate space.
\end{itemize}
The logical chain is:
\[
  \underbrace{z = re^{i\theta}}_{\text{pre-collapse (2D)}}
  \;\xrightarrow{\text{one collapse}}
  \underbrace{r\cos\theta\in\mathbb{R}}_{\text{one real value (1D)}}
  \;\xrightarrow{D\text{ independent collapses}}
  \underbrace{(x_1,\dots,x_D)\in\mathbb{R}^D}_{\text{emergent }D\text{D space.}}
\]
The $D=3$ of everyday experience is not written into EMTS by hand; it is the observed count of
independent mass-like degrees of freedom that decohere at macroscopic energy scales. Before any collapse
$r$ is merely a scalar scale, not a spatial coordinate; the full $D$-dimensional spatial arena only
crystallises through repeated projection. A purely wave-driven system --- one described by $(r,\theta)$
but never collapsed --- inhabits a 2D world without spatial extent in the particle sense
(Section~\ref{sec:wave-universe-dimensionality}).

\paragraph{Summary of the space-time duality.}
Putting the two identifications together:

\begin{center}
\begin{tabular}{lll}
\hline
\textbf{Polar coordinate} & \textbf{Physical meaning} & \textbf{Character} \\
\hline
$r = |z|$ & Spatial scale (inverse wavelength, Compton size, curvature) & Static; non-negative; sets the arena \\
$\theta = \omega t$ & Time (monotonically growing, but $2\pi$-periodic effect) & Dynamic; unbounded; drives evolution \\
\hline
\end{tabular}
\end{center}

A physical event is therefore fully characterised by a single complex number $z = r e^{i\theta}$:
$r$ says \emph{where} (at what spatial scale) and $\theta$ says \emph{when} (at what phase of the
evolution).  This is the EMTS replacement for the separate $(x, t)$ coordinates of Newtonian
mechanics — compressed into the magnitude and argument of a single complex amplitude.

\subsection{Measurement and collapse}

In standard quantum mechanics, measurement "collapses" the wavefunction. In the EMTS framework:
\begin{enumerate}
\item Before measurement: Full state is $z = m + iE$ with complex dynamics
\item Measurement: Projects to $\Re(z) = m$
\item After measurement: The imaginary component $E$ remains but is decohered from the observable $m$
\end{enumerate}

Multiple branches with different $E$ values persist along the imaginary axis (see Chapter~\ref{ch:mapping-reality} for the multiverse interpretation).

\section{Force quadrants}
\label{sec:force-quadrants}

The complex plane is divided into four quadrants based on the sign of $m$ and $E$. We hypothesize that fundamental forces correspond to phase windows:

\begin{center}
\begin{tabular}{clcc}
\hline
\textbf{Quadrant} & \textbf{Force} & \textbf{Phase Range} & \textbf{Symbol} \\
\hline
I & Electromagnetic & $0 < \theta < \frac{\pi}{2}$ & \quadEM \\
II & Weak & $\frac{\pi}{2} < \theta < \pi$ & \quadWeak \\
III & Strong & $\pi < \theta < \frac{3\pi}{2}$ & \quadStrong \\
IV & Gravitational & $\frac{3\pi}{2} < \theta < 2\pi$ & \quadGrav \\
\hline
\end{tabular}
\end{center}

Each force is characterized by a phase window $W_f(\theta)$ and coupling function $g_f(\theta, r)$. The observed interaction strength depends on where the state $z(t)$ resides in the phase cycle (see Chapter~\ref{ch:fundamental-forces} for details).

\section{Notation conventions}

Throughout this work, we use the following notation:

\begin{itemize}
\item $z = m + iE = r e^{i\theta}$ — Complex state (canonical form: $\EMTSz$)
\item $\theta = \omega t$ — Phase-time (unwrapped $\in\mathbb{R}$; cyclic mod $2\pi$; shorthand: $\phaseTime$)
\item $r = |z| = \sqrt{m^2+E^2}$ — Modulus: spatial scale / inverse wavelength (shorthand: $r$)
\item $\Re(z) = r\cos\theta = \projReal$ — Real projection (observable mass)
\item $\Psi(z, t)$ or $\Psi(r, \theta)$ — Wavefunction on complex state space
\item $Q_{\text{I}}, Q_{\text{II}}, Q_{\text{III}}, Q_{\text{IV}}$ — Force quadrants (EM, Weak, Strong, Gravitational)
\end{itemize}

\section{Relationship to standard physics}

The EMTS framework is not a replacement for quantum mechanics or relativity, but a \emph{geometric reinterpretation} that:

\begin{itemize}
\item Recovers the Schrödinger equation in appropriate limits
\item Naturally incorporates special relativity through $r = \sqrt{m^2 + E^2}$
\item Provides a unified language for discussing particles, forces, and spacetime
\item Suggests extensions (e.g., multiverse structure, dark matter) that emerge from the geometry
\end{itemize}

Standard formulations are recovered by:
\begin{enumerate}
\item Restricting to the real axis (classical limit)
\item Quantizing the phase $\theta$ (quantum mechanics)
\item Coupling the modulus $r$ to spacetime curvature (general relativity)
\end{enumerate}

\section{Summary: The EMTS postulates}

We summarize the core framework:

\begin{enumerate}
\item \textbf{Complex representation}: Physical states are complex numbers $z = m + iE$.
\item \textbf{Space as modulus}: The modulus $r = |z|$ encodes the spatial scale of the state
      (inverse de~Broglie wavelength, Compton size, gravitational reach).
      See Section~\ref{sec:space-as-modulus}.
\item \textbf{Time as phase}: The argument $\theta = \omega t$ encodes time.  It grows
      \emph{monotonically} (arrow of time) but acts \emph{cyclically} modulo $2\pi$ (wave
      nature of matter).  See Section~\ref{sec:time-as-phase}.
\item \textbf{Polar dynamics}: States evolve as $z(t) = r e^{i\omega t}$ with $\omega = E/\hbar$.
\item \textbf{Projection principle}: Observable reality is $\Re(z) = r\cos\theta$;
      imaginary components are hidden but dynamically active.
\item \textbf{Phase-force mapping}: Force channel is tagged by the quadrant of
      $\theta \bmod 2\pi$.
\item \textbf{Quantization}: Single-valuedness under $\theta \to \theta + 2\pi$ forces
      integer winding numbers and discrete spectra.
\item \textbf{Dimensionality from precision}: Effective spatial dimensionality increases with
      measurement precision (Chapter~\ref{ch:mapping-reality}).
\end{enumerate}

The remaining chapters develop these postulates in detail, exploring their implications for quantum mechanics (Part I), the Standard Model (Part II), specific phenomena (Part III), and future directions (Part IV).
