% Part I — Foundations
\chapter{The EMTS Framework: Core Definitions}
\label{ch:emts-framework}

Having introduced the complex plane and its historical significance, we now establish the foundational elements of the Energy-Mass-Time-Space (EMTS) framework. This chapter consolidates the core mathematical definitions and physical interpretations that underpin the entire theory.

\section{The complex plane mapping}
\label{sec:complex-mapping}

The EMTS framework posits that physical states can be represented as points in the complex plane:
\[
z = m + iE,
\]
where:
\begin{itemize}
\item $m$ is the mass component (mapped to the real axis)
\item $E$ is the energy component (mapped to the imaginary axis)
\item $i = \sqrt{-1}$ is the imaginary unit
\end{itemize}

This choice reflects the intuition that mass is "more tangible" than energy in everyday experience, though the mapping could be reversed without loss of generality. What matters is the \emph{relationship} between the components, not which axis they occupy.

\section{Polar representation}
\label{sec:polar-rep}

Every complex point $z$ can be expressed in polar form:
\[
z = r e^{i\theta} = r(\cos\theta + i\sin\theta),
\]
where:
\begin{itemize}
\item $r = |z| = \sqrt{m^2 + E^2}$ is the \textbf{modulus} (unified magnitude)
\item $\theta = \arg(z) = \arctan(E/m)$ is the \textbf{argument} (unified phase)
\end{itemize}

\subsection{Physical interpretation of $r$ and $\theta$}

The modulus $r$ represents the total mass-energy content:
\[
r = \sqrt{(mc^2)^2 + E^2} \quad \text{or in natural units} \quad r = \sqrt{m^2 + E^2}.
\]

The argument $\theta$ encodes the \textbf{partition} between mass and energy:
\begin{align}
m &= r\cos\theta, \\
E &= r\sin\theta.
\end{align}

Crucially, we identify $\theta$ with time evolution:
\[
\theta(t) = \omega t,
\]
where $\omega$ is the angular frequency related to energy by $E = \hbar\omega$ (Planck's quantum hypothesis).

\subsection{The projection principle}
\label{sec:projection-principle}

\textbf{Central postulate:} Physical reality corresponds to the \emph{projection} of the complex state onto the real axis:
\[
\text{Observable reality} = \Re(z) = m = r\cos\theta.
\]

This projection is \textbf{many-to-one}: infinitely many complex states
\[
\{z_n = m + iE_n\}_{n}
\]
all project to the same observable value $m$. The imaginary component $E$ represents hidden degrees of freedom that influence dynamics but are not directly observed.

\section{Fundamental relationships}
\label{sec:fundamental-relations}

\subsection{Time as phase}

The identification $\theta = \omega t$ unifies time evolution with rotation in the complex plane. A state evolves as:
\[
z(t) = r e^{i\omega t}.
\]

The observable projection oscillates:
\[
m(t) = r\cos(\omega t),
\]
explaining wave-like behavior from the geometry of rotation.

\subsection{Energy-frequency connection}

From quantum mechanics:
\[
E = \hbar\omega \quad \Rightarrow \quad \omega = \frac{E}{\hbar}.
\]

Thus the rotation rate in the complex plane is proportional to energy, linking dynamics to the imaginary component.

\subsection{Measurement and collapse}

In standard quantum mechanics, measurement "collapses" the wavefunction. In the EMTS framework:
\begin{enumerate}
\item Before measurement: Full state is $z = m + iE$ with complex dynamics
\item Measurement: Projects to $\Re(z) = m$
\item After measurement: The imaginary component $E$ remains but is decohered from the observable $m$
\end{enumerate}

Multiple branches with different $E$ values persist along the imaginary axis (see Chapter~\ref{ch:mapping-reality} for the multiverse interpretation).

\section{Force quadrants}
\label{sec:force-quadrants}

The complex plane is divided into four quadrants based on the sign of $m$ and $E$. We hypothesize that fundamental forces correspond to phase windows:

\begin{center}
\begin{tabular}{clcc}
\hline
\textbf{Quadrant} & \textbf{Force} & \textbf{Phase Range} & \textbf{Symbol} \\
\hline
I & Electromagnetic & $0 < \theta < \frac{\pi}{2}$ & \quadEM \\
II & Weak & $\frac{\pi}{2} < \theta < \pi$ & \quadWeak \\
III & Strong & $\pi < \theta < \frac{3\pi}{2}$ & \quadStrong \\
IV & Gravitational & $\frac{3\pi}{2} < \theta < 2\pi$ & \quadGrav \\
\hline
\end{tabular}
\end{center}

Each force is characterized by a phase window $W_f(\theta)$ and coupling function $g_f(\theta, r)$. The observed interaction strength depends on where the state $z(t)$ resides in the phase cycle (see Chapter~\ref{ch:fundamental-forces} for details).

\section{Notation conventions}

Throughout this work, we use the following notation:

\begin{itemize}
\item $z = m + iE = r e^{i\theta}$ — Complex state (canonical form: $\EMTSz$)
\item $\theta = \omega t$ — Phase-time relation (shorthand: $\phaseTime$)
\item $r = |z|$ — Modulus (total mass-energy)
\item $\Re(z) = r\cos\theta = \projReal$ — Real projection (observable)
\item $\Psi(z, t)$ or $\Psi(r, \theta)$ — Wavefunction on complex state space
\item $Q_{\text{I}}, Q_{\text{II}}, Q_{\text{III}}, Q_{\text{IV}}$ — Force quadrants (EM, Weak, Strong, Gravitational)
\end{itemize}

\section{Relationship to standard physics}

The EMTS framework is not a replacement for quantum mechanics or relativity, but a \emph{geometric reinterpretation} that:

\begin{itemize}
\item Recovers the Schrödinger equation in appropriate limits
\item Naturally incorporates special relativity through $r = \sqrt{m^2 + E^2}$
\item Provides a unified language for discussing particles, forces, and spacetime
\item Suggests extensions (e.g., multiverse structure, dark matter) that emerge from the geometry
\end{itemize}

Standard formulations are recovered by:
\begin{enumerate}
\item Restricting to the real axis (classical limit)
\item Quantizing the phase $\theta$ (quantum mechanics)
\item Coupling the modulus $r$ to spacetime curvature (general relativity)
\end{enumerate}

\section{Summary: The EMTS postulates}

We summarize the core framework:

\begin{enumerate}
\item \textbf{Complex representation}: Physical states are complex numbers $z = m + iE$
\item \textbf{Polar dynamics}: States evolve as $z(t) = r e^{i\omega t}$ with $\omega = E/\hbar$
\item \textbf{Projection principle}: Observable reality is $\Re(z)$; imaginary components are hidden
\item \textbf{Phase-force mapping}: Forces correspond to phase windows in $[0, 2\pi)$
\item \textbf{Quantization}: Periodicity in $\theta$ leads to discrete spectra
\item \textbf{Dimensionality from precision}: Effective dimensionality increases with measurement precision
\end{enumerate}

The remaining chapters develop these postulates in detail, exploring their implications for quantum mechanics (Part I), the Standard Model (Part II), specific phenomena (Part III), and future directions (Part IV).
