\chapter{The Quadrants and Their Implications}
\label{ch:fundamental-forces}

The EMTS framework assigns a physical meaning not just to the magnitude $r$ and the phase $\theta$
of the complex state $\EMTSz$, but to the \emph{quadrant} in which the state resides at the moment
of interaction. The central hypothesis of this chapter is:

\begin{quote}
\textbf{Force-tagging postulate.}
The interaction channel observed in spacetime is tagged by the quadrant of the pre-collapse state
$z(t_0) = r e^{i\theta_0}$. Each of the four fundamental forces is the dominant interaction mode
for states residing in the corresponding quarter of the complex $m$--$E$ plane.
\end{quote}

This is not merely a labelling convenience. The signs of $m=r\cos\theta_0$ and $E=r\sin\theta_0$
encode qualitatively different physical regimes—positive vs.\ negative effective mass, absorptive
vs.\ emissive energy—and each regime selects a different symmetry group and mediating boson family.

% -----------------------------------------------------------------------
\section{Phase Windows as Hermitian Projectors}
\label{sec:phase-windows}
% -----------------------------------------------------------------------

The mathematical object that carves out a quadrant is a \textbf{phase-window projector}.
Define the indicator function for force $f$:
\[
  W_f(\theta)
  = \begin{cases}
    1, & \theta \in \Delta_f, \\
    0, & \text{otherwise,}
  \end{cases}
\]
where $\Delta_f$ is the phase interval assigned to force $f$ (Table~\ref{tab:force-quadrants}).
A smooth, physical version replaces the step function with a squared cosine bell
\[
  \tilde{W}_f(\theta)
  = \cos^2\!\left(\frac{\theta - \theta_f^{\rm mid}}{\Delta\theta_f}\,\frac{\pi}{2}\right)
    \cdot \mathbf{1}_{\theta\in\Delta_f}
\]
that tapers to zero at the quadrant boundaries, avoiding divergences in the coupling
integral. Coupling strength at scale $r$ is then the cycle average
\[
  \bar{g}_f(r)
  = \frac{\omega(r)}{2\pi}
    \int_0^{2\pi} W_f(\theta)\,g_f(\theta, r)\,d\theta,
\]
where $g_f(\theta, r)$ is a force-specific coupling function discussed in each subsection
below. In the Hilbert-space language of Chapter~\ref{ch:hermitian-operators}, $W_f$
acts as a Hermitian projector onto the subspace of states whose phase lies in $\Delta_f$,
and $\bar{g}_f$ is the expectation value of the coupling operator in that subspace.

\begin{table}[h]
\centering
\begin{tabular}{clcccc}
\hline
\textbf{Quadrant} & \textbf{Force} & \textbf{Phase window $\Delta_f$} & $\text{sgn}(m)$ & $\text{sgn}(E)$ & \textbf{Mediator} \\
\hline
\quadEM           & Electromagnetic & $\left(0,\;\tfrac{\pi}{2}\right)$             & $+$ & $+$ & Photon $\gamma$ \\
\quadWeak         & Weak nuclear    & $\left(\tfrac{\pi}{2},\;\pi\right)$           & $-$ & $+$ & $W^\pm,\;Z^0$ \\
\quadStrong       & Strong nuclear  & $\left(\pi,\;\tfrac{3\pi}{2}\right)$          & $-$ & $-$ & Gluons $g$ \\
\quadGrav         & Gravitational   & $\left(\tfrac{3\pi}{2},\;2\pi\right)$         & $+$ & $-$ & Graviton $G$ \\
\hline
\end{tabular}
\caption{Assignment of fundamental forces to phase quadrants of the EMTS complex plane.
The signs of the real projection $m=r\cos\theta$ and imaginary component $E=r\sin\theta$
characterise each regime.}
\label{tab:force-quadrants}
\end{table}

% -----------------------------------------------------------------------
\section{Quadrant I — Electromagnetic Force}
\label{sec:quadrant-em}
% -----------------------------------------------------------------------

\textbf{Phase window:} $\theta \in \left(0,\tfrac{\pi}{2}\right)$, \quad
$m = r\cos\theta > 0$, \quad $E = r\sin\theta > 0$.

In \quadEM\ the system possesses \emph{positive real mass and positive energy}. This is
the most familiar physical regime: a massive, energetic particle propagating with a
well-defined positive rest mass.

\subsection*{Physical character}

The electromagnetic force is mediated by the massless photon, which corresponds to the
\emph{boundary} $\theta = \pi/2$ of \quadEM—the pure-imaginary axis where $m = 0$ and
the state carries only energy. Because the mediator lives on the quadrant boundary, the
force is infinite-ranged: the photon can transport energy $E = \hbar\omega$ across
arbitrarily large distances without a mass penalty.

Within \quadEM, the coupling function is approximately
\[
  g_{\rm EM}(\theta, r) \approx \frac{\alpha}{r}\,\cos^2\theta,
  \qquad
  \alpha \approx \frac{1}{137},
\]
where $\alpha$ is the fine-structure constant. The $\cos^2\theta$ factor reflects the
projection principle: only the mass component $m/r = \cos\theta$ couples to the
electromagnetic field. Purely energetic states ($\theta \to \pi/2$) carry less charge
and interact more weakly, while purely massive states ($\theta \to 0$) couple most
strongly.

\subsection*{Symmetry in EMTS}

The gauge symmetry of electromagnetism, $U(1)$, corresponds to a global phase shift
$\theta \mapsto \theta + \alpha$ that leaves $r$ unchanged. In the Hilbert-space
representation this is the unitary operator
$U(\alpha) = e^{i\alpha\hat{N}}$,
where $\hat{N} = \lvert E\rangle\langle E\rvert - \lvert m\rangle\langle m\rvert$
counts the energy-to-mass imbalance. Electromagnetic interactions are thus phase
rotations that keep the system within (or near) \quadEM.

% -----------------------------------------------------------------------
\section{Quadrant II — Weak Nuclear Force}
\label{sec:quadrant-weak}
% -----------------------------------------------------------------------

\textbf{Phase window:} $\theta \in \left(\tfrac{\pi}{2}, \pi\right)$, \quad
$m = r\cos\theta < 0$, \quad $E = r\sin\theta > 0$.

In \quadWeak\ the real projection $m$ is \emph{negative} while the energy component
remains positive. Negative $m$ does not imply a physical anti-particle; rather, it signals
that the state has crossed the pure-energy axis and the observable on the real axis now
points in the negative direction—it is an \emph{inverted mass configuration}.

\subsection*{Physical character}

The weak force is mediated by the massive $W^\pm$ and $Z^0$ bosons. Their masses
($m_W \approx 80\;\text{GeV}$, $m_Z \approx 91\;\text{GeV}$) arise because the mediators
\emph{must} cross from the \quadWeak\ regime back toward the real axis to transmit the
interaction, paying a mass penalty encoded by the crossing distance $|\theta - \pi/2|$.
This gives the weak force its characteristically short range,
\[
  \lambda_{\rm weak} \sim \frac{\hbar c}{m_W c^2} \approx 2 \times 10^{-18}\;\text{m}.
\]

\subsection*{Flavour mixing and the EMTS picture}

A hallmark of the weak force is \textbf{flavour mixing}: it transforms one quark or
lepton flavour into another. In EMTS terms, a weak interaction shifts the phase $\theta$
discontinuously across the \quadEM--\quadWeak\ boundary ($\theta = \pi/2$), moving the
state from a positive-mass to a negative-mass configuration and back. This boundary
transition corresponds to a \emph{parity flip} ($m \to -m$), naturally explaining why
the weak force is the only interaction that violates parity ($P$) symmetry—it is the
only force whose mediator must cross the imaginary axis.

The coupling function is approximately
\[
  g_{\rm W}(\theta, r) \approx \frac{g_2^2}{r}\,e^{-m_W r/\hbar c}\,|\cos\theta|,
\]
where $g_2$ is the weak coupling constant and the exponential Yukawa suppression
reflects the massive mediator.

% -----------------------------------------------------------------------
\section{Quadrant III — Strong Nuclear Force}
\label{sec:quadrant-strong}
% -----------------------------------------------------------------------

\textbf{Phase window:} $\theta \in \left(\pi, \tfrac{3\pi}{2}\right)$, \quad
$m = r\cos\theta < 0$, \quad $E = r\sin\theta < 0$.

\quadStrong\ is the only quadrant where \emph{both} components are negative. This
doubly-inverted regime corresponds to states deep inside nuclear scales, where both the
observable mass projection and the energy carrying capacity point opposite to the
macroscopic conventions.

\subsection*{Physical character}

The strong force is mediated by eight massless gluons and governs quark confinement.
Its key distinguishing feature is \textbf{asymptotic freedom} and \textbf{confinement}:

\begin{itemize}
  \item \textbf{Asymptotic freedom} ($r \to 0$): At short distances the coupling weakens.
        In EMTS, small $r$ means the state has low modulus (low mass-energy), and both
        negative components are small in magnitude—the state barely occupies \quadStrong\ and
        couples weakly.
  \item \textbf{Confinement} ($r \to \infty$): As $r$ grows, states deep in \quadStrong\
        cannot close their EMTS orbit into a real projection. The coupling grows linearly:
        \[
          g_{\rm S}(\theta, r) \approx \alpha_s(r)\,r,
          \qquad
          \alpha_s(r) \sim \frac{1}{\ln(r_0/r)},
        \]
        producing a flux tube of rising potential that prevents isolated quarks from
        appearing as free observable particles.
\end{itemize}

\subsection*{Colour charge and SU(3)}

The three colour charges (red, green, blue) correspond to three independent phase
orientations within \quadStrong, each separated by $2\pi/3$. Only colour-neutral
combinations—states whose joint phase averages to $\pi$ (the midpoint of \quadStrong, i.e.,
on the negative real axis)—produce closed EMTS orbits that project to a real, observable
hadron. This is the EMTS geometric interpretation of $SU(3)$ colour neutrality.

% -----------------------------------------------------------------------
\section{Quadrant IV — Gravitational Force}
\label{sec:quadrant-grav}
% -----------------------------------------------------------------------

\textbf{Phase window:} $\theta \in \left(\tfrac{3\pi}{2}, 2\pi\right)$, \quad
$m = r\cos\theta > 0$, \quad $E = r\sin\theta < 0$.

In \quadGrav\ the mass projection is again positive—as in \quadEM—but the energy
component is \emph{negative}. A negative imaginary component means the state is
decelerating, emitting energy, or losing phase coherence: the hallmarks of
gravitational attraction and energy dissipation.

\subsection*{Physical character}

Gravity is the weakest force at particle scales but the only one that is always
attractive and infinite-ranged. Its mediator, the graviton $G$, is conjectured to be
massless (like the photon), and it lives on the \quadGrav--\quadEM\ boundary at
$\theta = 0$ (or equivalently $2\pi$)—the positive real axis, where pure mass and zero
energy meet.

The coupling function is
\[
  g_{\rm G}(\theta, r) \approx \frac{G_N\,r^2}{r_{\rm P}^2}\,\cos^2\theta,
  \qquad
  r_{\rm P} = \sqrt{\frac{\hbar G_N}{c^3}} \approx 1.6 \times 10^{-35}\;\text{m},
\]
where $G_N$ is Newton's constant and $r_{\rm P}$ is the Planck length. The $\cos^2\theta$
factor is the same as in electromagnetism—both forces couple to the mass projection—but
the scale factor $r^2/r_{\rm P}^2$ suppresses gravity enormously at sub-Planckian
radii, explaining the hierarchy problem geometrically: gravity is weak because the
Planck length is tiny compared to every other scale in the theory.

\subsection*{Always attractive}

In \quadGrav\ we have $E < 0$, meaning the state is losing energy to the field. Since
the force is proportional to $-\partial g_{\rm G}/\partial r \propto -G_N m_1 m_2/r^2$
(always negative, i.e., attractive), gravity never repels. There is no quadrant
analogous to the electromagnetic case where $E > 0$ states could push back, so the
force has no repulsive branch. This geometric asymmetry between \quadEM\ and \quadGrav\
is the EMTS explanation for why gravity cannot be screened.

% -----------------------------------------------------------------------
\section{Boundary Axes as Transition Zones}
\label{sec:boundary-axes}
% -----------------------------------------------------------------------

The four quadrant boundaries (the coordinate axes of the $m$--$E$ plane) are
physically significant in their own right:

\begin{center}
\begin{tabular}{clll}
\hline
\textbf{Axis} & \textbf{Condition} & \textbf{State type} & \textbf{Physical role} \\
\hline
$\theta = 0$        & $m > 0,\;E = 0$ & Pure mass (rest)    & Classical matter; graviton pole \\
$\theta = \pi/2$    & $m = 0,\;E > 0$ & Pure energy         & Massless bosons; photon pole \\
$\theta = \pi$      & $m < 0,\;E = 0$ & Anti-mass           & Antimatter boundary; $CP$ axis \\
$\theta = 3\pi/2$   & $m = 0,\;E < 0$ & Negative energy     & Vacuum fluctuations; Casimir states \\
\hline
\end{tabular}
\end{center}

States sitting exactly on an axis are eigenstates of either $\hat{M}$ or $\hat{E}_{\rm
obs}$ (Chapter~\ref{ch:hermitian-operators}), and the axes themselves act as
\emph{force-transition interfaces}. A particle undergoing a phase transition from one
quadrant to another must necessarily pass through one of these axes, which is why
interactions at the quadrant boundaries are associated with the emission or absorption
of the corresponding massless (boundary-resident) mediator.

% -----------------------------------------------------------------------
\section{Force Hierarchy from Phase Geometry}
\label{sec:force-hierarchy}
% -----------------------------------------------------------------------

A qualitative but illuminating observation emerges when the four cycle-averaged coupling
strengths are compared as a function of $r$. Each $\bar{g}_f(r)$ inherits a different
$r$-dependence from the structure of its quadrant:

\begin{center}
\begin{tabular}{lll}
\hline
\textbf{Force} & $r$-scaling of $\bar{g}_f$ & \textbf{Observed range} \\
\hline
Strong         & $\sim \alpha_s(r)\cdot r$  & $\lesssim 10^{-15}$ m (grows then confines) \\
Electromagnetic& $\sim \alpha / r$          & $\infty$ (power-law) \\
Weak           & $\sim g_2^2 e^{-m_W r}/r$ & $\sim 10^{-18}$ m (Yukawa) \\
Gravitational  & $\sim G_N r^2/r_{\rm P}^2$& $\infty$ (power-law, tiny $G_N$) \\
\hline
\end{tabular}
\end{center}

At the scale of everyday matter ($r \sim r_{\rm Bohr} \approx 0.5\;\text{\AA}$), the
hierarchy $\alpha_s \gg \alpha \gg g_2^2 e^{-m_W r} \gg G_N/r_{\rm P}^2 r^{-2}$ is
entirely consistent with the standard measured values. The EMTS quadrant model does not
derive these numerical values from first principles, but it \emph{organises} them into a
coherent geometric structure: forces become arrayed in the complex plane not by
coincidence but because each quadrant imposes different sign constraints and range
behaviours on the coupling integral.

% -----------------------------------------------------------------------
\section{Is the Four-Fold Coincidence Real?}
\label{sec:four-fold}
% -----------------------------------------------------------------------

A natural—and important—question arises at this point:

\begin{quote}
\emph{Is it a coincidence that there are four fundamental forces and four quadrants?}
\end{quote}

The short answer, within the EMTS framework, is \textbf{no}. But the reasoning deserves
care, because the same count "four'' carries different epistemic weights on each side of
the equation.

\subsection*{Why four quadrants is inevitable}

The complex plane $\mathbb{C}$ has two real coordinates, $m$ and $E$, each of which
can be positive or negative. The number of sign combinations is
\[
  |\{+,-\}|^2 = 2^2 = 4.
\]
This is not a hypothesis about physics—it is an arithmetic fact about the dimensionality
of the representation space. Given the EMTS postulate that physical states are complex
numbers $z = m + iE$, \emph{there could not be any other number of quadrants}.

More formally, the symmetry group that permutes sign combinations is the Klein four-group
$V_4 \cong \mathbb{Z}_2 \times \mathbb{Z}_2$, generated by the two independent reflections
\[
  \sigma_m : m \mapsto -m \quad (\text{mass parity}),
  \qquad
  \sigma_E : E \mapsto -E \quad (\text{energy parity}).
\]
The four quadrants are exactly the four orbits of $V_4$ acting on the non-zero points of
$\mathbb{C}\setminus\{0\}$. Any two-dimensional real representation of a physical state
will have this fourfold orbit structure.

\subsection*{Why four forces is \emph{not} fundamental}

From the standard-physics side, the count "four'' is historically and energetically
contingent:

\begin{itemize}
  \item \textbf{Electroweak unification} (Glashow--Salam--Weinberg, 1960s--70s): at
        energies above $\sim 100\;\text{GeV}$ the electromagnetic and weak forces merge
        into a single $SU(2)_L \times U(1)_Y$ interaction. The apparent split into two
        forces at low energies arises from spontaneous symmetry breaking (the Higgs
        mechanism), which gives the $W^\pm$ and $Z^0$ bosons their masses and leaves the
        photon massless. If one works above the electroweak scale, there are effectively
        \emph{three} forces, not four.
  \item \textbf{Grand Unified Theories (GUTs)} predict a further merger of the
        electroweak and strong forces into a single $SU(5)$ (or larger) gauge group at
        energies $\sim 10^{15}\;\text{GeV}$, reducing the count to \emph{two}: a single
        unified force plus gravity.
  \item \textbf{String/M-theory} candidates suggest all four may be facets of a single
        eleven-dimensional geometric structure, making the count \emph{one}.
\end{itemize}

So the number four is a low-energy, broken-symmetry artefact of the Standard Model, not
a Platonic constant of nature.

\subsection*{The EMTS interpretation}

Putting both sides together, the EMTS perspective is the following:

\begin{enumerate}
  \item The complex plane \emph{necessarily} produces exactly four sign-distinct regimes.
  \item Nature at low energies \emph{empirically} exhibits four dominant interaction
        channels.
  \item The EMTS framework is a mapping of the low-energy effective theory: it assigns
        each empirically observed force to the quadrant whose sign structure best matches
        the force's qualitative properties (range, attractive vs.\ repulsive, mass-dependent
        vs.\ charge-dependent coupling).
\end{enumerate}

The match between four quadrants and four forces is therefore a \emph{necessary}
consequence of EMTS applied to the low-energy Standard Model, not a numerical
coincidence. The deeper question—whether the fourfold structure of the complex plane is
the \emph{reason} nature exhibits (approximately) four forces at low energies, or merely
a convenient coordinate system for describing them—remains a topic of ongoing
investigation in the theoretical framework.

\subsection*{Higher-energy modifications}

If the electroweak unification is taken seriously within EMTS, then above the
electroweak scale the \quadEM\ and \quadWeak\ quadrants must \emph{merge}. This would
correspond to the phase boundary at $\theta = \pi/2$ becoming dynamical: the imaginary
axis ceases to be a sharp dividing wall between EM and Weak regimes and instead becomes
a transition band of width $\sim \hbar/m_W c$ that widens as energy increases. At GUT
scales, three quadrants merge further, and one would work with an effective single-quadrant
covering almost the entire upper half-plane
$\{\theta \in (0, \pi)\}$. The geometrical language survives high-energy unification; only
the sharpness of the quadrant boundaries changes.

% -----------------------------------------------------------------------
\section{Quadrant Transitions: Speculation on Induced Force-Channel Switching}
\label{sec:quadrant-transitions}
% -----------------------------------------------------------------------

\begin{quote}
\emph{What happens if a system is driven from one quadrant to another?}
\end{quote}

Under natural evolution the phase $\theta(t) = \omega t$ advances continuously, so a
freely evolving EMTS state sweeps through all four quadrants in one period $T =
2\pi/\omega$. This is ordinary, unforced dynamics. The interesting question is what it
would mean to \emph{engineer} a transition—to apply an external perturbation that
moves the state from its natural quadrant into a different one, changing its dominant
interaction channel.

% -----------------------------------------------------------------------
\subsection{Boundary crossings and their cost}
\label{ssec:boundary-cost}
% -----------------------------------------------------------------------

Each quadrant boundary is a specific coordinate axis of the $m$--$E$ plane. Crossing it
requires passing through a qualitatively special intermediate state:

\begin{center}
\begin{tabular}{clll}
\hline
\textbf{Boundary} & \textbf{Condition at crossing} & \textbf{Intermediate state} & \textbf{Energy cost} \\
\hline
$\theta = \pi/2$   & $m = 0,\; E > 0$ & Momentarily massless, energetic &
  $\Delta E \sim m_W c^2 \approx 80\;\text{GeV}$ \\
$\theta = \pi$     & $m < 0,\; E = 0$ & Anti-mass, zero energy &
  $\Delta E \sim \Lambda_{\rm QCD} \approx 200\;\text{MeV}$ \\
$\theta = 3\pi/2$  & $m = 0,\; E < 0$ & Massless, negative energy &
  $\Delta E \sim E_{\rm P} = \sqrt{\hbar c^5/G_N}$ (Planck scale) \\
$\theta = 0\,(2\pi)$ & $m > 0,\; E = 0$ & Pure rest mass, zero kinetic energy &
  $\Delta E \sim 0$ (free, any mass) \\
\hline
\end{tabular}
\end{center}

The $\theta = 3\pi/2$ crossing (entering or leaving \quadGrav) demands a Planck-scale
energy to sustain a massless, negative-energy intermediate state—which is why the
gravitational force is not only weak but effectively \emph{non-switchable} at any
energy accessible today. The $\theta = 0$ boundary, by contrast, is free: a particle at
rest with positive mass naturally sits there, and the cost of slightly perturbing $\theta$
away from zero is $O(\omega)$—achievable with any nonzero energy input.

% -----------------------------------------------------------------------
\subsection{The six transitions: a taxonomy}
\label{ssec:transition-taxonomy}
% -----------------------------------------------------------------------

There are six distinct pairs of quadrants: four \emph{adjacent} (sharing a boundary
axis) and two \emph{diagonal} (no shared boundary). Table~\ref{tab:transitions}
catalogues them.

\begin{table}[h]
\centering
\begin{tabular}{llllc}
\hline
\textbf{Transition} & \textbf{Axis crossed} & \textbf{Sign change} & \textbf{Physical interpretation} & \textbf{Known analogue} \\
\hline
\quadEM$\to$\quadWeak   & $\theta=\pi/2$     & $m: +\to-$          & Mass becomes anti-mass; parity flip        & $\beta$-decay, weak CC \\
\quadWeak$\to$\quadStrong & $\theta=\pi$    & $E: +\to-$          & Energy-absorbing becomes energy-emitting   & Hadronisation \\
\quadStrong$\to$\quadGrav & $\theta=3\pi/2$ & $m: -\to+$          & Confinement released into gravity          & Speculative \\
\quadGrav$\to$\quadEM     & $\theta=0$      & $E: -\to+$          & Gravitational decay into EM radiation      & Hawking radiation \\
\hline
\quadEM$\leftrightarrow$\quadStrong & $\pi/2$ then $\pi$ & $m,E$ both flip  & Full sign inversion; matter$\leftrightarrow$anti-matter+confinement & Annihilation/pair prod.\ \\
\quadWeak$\leftrightarrow$\quadGrav & $\pi$ then $3\pi/2$& $m,E$ both flip  & Weak-sector anti-mass in gravitational collapse & Speculative \\
\hline
\end{tabular}
\caption{All six quadrant-pair transitions, the boundary axis crossed, the sign that
flips, and the closest known physical process.}
\label{tab:transitions}
\end{table}

% -----------------------------------------------------------------------
\subsection{Adjacent transitions in known physics}
\label{ssec:adjacent-known}
% -----------------------------------------------------------------------

Several standard processes can be re-read as quadrant transitions.

\paragraph{\quadEM$\,\to\,$\quadWeak\ (crossing $\theta=\pi/2$): Beta decay.}
In neutron beta decay, $n \to p + e^- + \bar\nu_e$, a down quark in the Strong/EM
regime acquires a $W^-$ vertex. In EMTS terms the quark's phase is pushed across
$\theta = \pi/2$: $m$ briefly vanishes (massless intermediate state mediated by $W^-$)
and then re-emerges negative. The proton is a composite that re-establishes a positive-$m$
projection by redistributing phase among its three quarks, while the electron carries
the crossed-phase information out as an observable particle. The parity violation
observed in weak decays is, in this picture, the direct signature of $m$ changing
sign—the observable projection $\Re(z)$ reverses under the crossing.

\paragraph{\quadGrav$\,\to\,$\quadEM\ (crossing $\theta=0$): Hawking radiation.}
A black hole's gravitational field corresponds to states deep in \quadGrav\ ($m>0$,
$E<0$, emitting energy inward). Hawking's calculation shows that quantum fluctuations
at the event horizon produce photons that escape to infinity. In EMTS this is a
\emph{boundary crossing at $\theta = 0$}: a virtual pair is produced straddling the
positive real axis, one partner falls deeper into \quadGrav\ (absorbed by the black
hole) and the other crosses into \quadEM\ ($E>0$) and propagates away as a real photon.
The boundary $\theta = 0$ is the cheapest crossing (zero rest-mass cost), which is
consistent with Hawking radiation being a low-energy, thermal process relative to the
Planck scale.

\paragraph{\quadWeak$\,\to\,$\quadStrong\ (crossing $\theta=\pi$): Hadronisation.}
After a high-energy weak collision produces free quarks, the strong force reasserts
itself and the quarks hadronise—forming bound states. In EMTS the quarks, briefly in
\quadWeak\ (high-energy, $E>0$, $m<0$), lose energy through gluon radiation and their
phase rotates through $\theta = \pi$ (the negative real axis, $E \to 0$ then $E < 0$),
settling into the confining \quadStrong\ regime. The $\Lambda_{\rm QCD}$ energy scale
of the boundary crossing matches the observed hadronisation energy scale
$\sim 200\;\text{MeV}$.

% -----------------------------------------------------------------------
\subsection{Diagonal transitions: simultaneous sign flips}
\label{ssec:diagonal-transitions}
% -----------------------------------------------------------------------

Diagonal transitions flip \emph{both} $\text{sgn}(m)$ and $\text{sgn}(E)$
simultaneously. In the operator language of Chapter~\ref{ch:hermitian-operators}, the
state passes through the origin $z = 0$—a singularity of undefined phase. There are
two diagonal pairs:

\paragraph{\quadEM$\,\leftrightarrow\,$\quadStrong.}
Both components flip: $m \to -m$, $E \to -E$. This is precisely the action of the
combined parity-times-time-reversal operator $PT$, which in quantum field theory maps a
particle to its \emph{CPT conjugate}. If \quadEM\ hosts ordinary matter (positive mass,
positive energy propagating forward in time) then \quadStrong\ contains, in this
interpretation, a shadow regime that is simultaneously anti-massed and time-reversed.
The process $e^+ + e^- \to \gamma\gamma$ (pair annihilation) can be read as the electron
($z \in $ \quadEM) and positron ($z \in $ \quadStrong\ by $PT$ symmetry) annihilating
at $z = 0$ and re-emerging as photons on the $\theta = \pi/2$ boundary.

\paragraph{\quadWeak$\,\leftrightarrow\,$\quadGrav.}
The other diagonal pair maps Weak states ($m<0$, $E>0$) to Gravitational states
($m>0$, $E<0$). This is speculative territory: no known process unambiguously
transitions between the weak and gravitational regimes. However, in the early universe
at temperatures above the electroweak scale, weak and gravitational interactions were
comparably strong. An EMTS perspective would suggest that at those energies the
\quadWeak--\quadGrav\ boundary becomes traversable—potentially observable as an
asymmetry between gravitational and weak contributions to the stress-energy tensor
in primordial cosmology.

% -----------------------------------------------------------------------
\subsection{Induced transitions: what would it take?}
\label{ssec:induced-transitions}
% -----------------------------------------------------------------------

Assuming we could engineer an external field that selectively shifts $\theta$, what
would "force-channel switching'' require in practice?

The transition operator that maps \quadEM\ into \quadWeak\ is $\sigma_m$, the mass
parity operator:
\[
  \sigma_m = \lvert m\rangle\langle m\rvert - \lvert E\rangle\langle E\rvert
  = \begin{pmatrix}1&0\\0&-1\end{pmatrix} = \sigma_z.
\]
Applied to the canonical EMTS state $\lvert\psi\rangle = \cos\theta\lvert m\rangle +
i\sin\theta\lvert E\rangle$, this gives
$\sigma_m\lvert\psi\rangle = \cos\theta\lvert m\rangle - i\sin\theta\lvert E\rangle$,
which is the state at phase $-\theta$ — a time-reversal in the EMTS picture. The
analogous energy parity, $\sigma_E = \sigma_x$ in the EMTS basis, reflects across the
real axis ($E \to -E$), mapping \quadEM\ to \quadGrav\ and \quadWeak\ to \quadStrong.
Their product
\[
  \sigma_m \sigma_E = \begin{pmatrix}0&1\\-1&0\end{pmatrix}
\]
performs the diagonal flip and corresponds to rotating by $\pi$ in the Hilbert space.

The practical requirement for each induced transition is therefore a Hamiltonian
perturbation $\delta\hat{H}$ that is \emph{off-diagonal} in the $\{\lvert m\rangle,
\lvert E\rangle\}$ basis — precisely the weak-force interaction Hamiltonian in the
Standard Model. This is not surprising: the purpose of the $W^\pm$ bosons is exactly to
move states across the mass-sign boundary.

\begin{center}
\begin{tabular}{lll}
\hline
\textbf{Induced transition} & \textbf{Required operator} & \textbf{Analogy} \\
\hline
\quadEM$\to$\quadWeak   & $\sigma_m = \sigma_z$       & $W^\pm$ vertex; parity-flip \\
\quadEM$\to$\quadGrav   & $\sigma_E = \sigma_x$       & Gravitational redshift; photon blue/redshift \\
\quadEM$\to$\quadStrong & $\sigma_m\sigma_E$           & Annihilation into gluons; QCD vacuum \\
\hline
\end{tabular}
\end{center}

% -----------------------------------------------------------------------
\subsection{Speculative implications}
\label{ssec:speculative-implications}
% -----------------------------------------------------------------------

Taking the quadrant-transition picture seriously leads to several speculative but
internally consistent hypotheses.

\paragraph{Force transmutation.}
If a macroscopic object could be driven into \quadStrong\ en masse — its constituent
particles' phases all shifted past $\theta = \pi$ simultaneously — the dominant
interaction would switch from electromagnetic to strong. The object would "colour up":
its atoms would no longer interact electromagnetically but would instead experience the
short-range, confining strong force. This is obviously prevented by the $\sim 80$ GeV
per-particle cost of crossing $\theta = \pi/2$ first, but it suggests that at
sufficiently high temperatures (above the QCD crossover, $T\gtrsim 150\;\text{MeV}$)
dense nuclear matter does indeed transition collectively into a quark-gluon plasma—a
state naturally residing in \quadStrong.

\paragraph{Gravitational-to-electromagnetic phase transition.}
The $\theta = 0$ boundary is the cheapest crossing. A speculative implication is that
a gravitationally-dominated object (a neutron star or black hole in \quadGrav) could
undergo a spontaneous phase transition into an electromagnetically-dominated state by
radiating energy ($E: - \to +$), essentially converting gravitational binding energy
into photons. Hawking radiation is a quantum-mechanical trace of this process. A fully
classical version may be relevant to the physics of magnetars, whose intense magnetic
fields suggest a large electromagnetic component co-existing with extreme gravitational
binding.

\paragraph{A traversable \quadStrong--\quadGrav\ corridor?}
The $\theta = 3\pi/2$ boundary normally requires Planck-scale energy. However, if the
modulus $r$ is very small — near the Planck length $r_P$ — the energy cost of the
crossing scales as $r/r_P$ and becomes of order unity in natural units. This suggests
that at the Planck scale, the boundary between the gravitational and strong quadrants
is no longer a barrier: states can fluctuate freely between \quadStrong\ and \quadGrav.
This is consistent with the popular conjecture that quantum gravity and the strong force
unify at the Planck scale, and it provides a geometric EMTS interpretation of that
unification: the confinement boundary and the gravitational boundary dissolve into each
other when $r \sim r_P$.

\paragraph{Complete quadrant cycling as a model of particle generation.}
If a state sweeps through all four quadrants in a single period $T = 2\pi/\omega$, it
experiences each force in sequence. An object that completes $n$ such cycles before
being observed will have accumulated a winding number $n$ around the origin. EMTS
speculates that this winding number corresponds to the \emph{generation number} of
the particle: first generation (electron, up, down) completes one clean cycle;
second generation (muon, strange, charm) has a half-integer additional winding from a
sub-cycle resonance; third generation (tau, bottom, top) winds twice before closing.
This offers a geometric origin for the three-generation structure of the Standard
Model — though the quantitative details remain to be worked out.

% -----------------------------------------------------------------------
\section{Summary}
\label{sec:forces-summary}
% -----------------------------------------------------------------------

The fundamental forces are distributed across the EMTS complex plane as follows:
\begin{enumerate}
  \item \textbf{\quadEM\ (Electromagnetic):} $m>0$, $E>0$. Long-range, massless
        photon mediator on the \quadEM--\quadWeak\ boundary. $U(1)$ phase symmetry.
        Coupling $\propto \cos^2\theta/r$.

  \item \textbf{\quadWeak\ (Weak):} $m<0$, $E>0$. Short-range, massive $W^\pm/Z^0$
        mediators. Parity-violating because its mediators must cross the imaginary axis.
        Coupling with Yukawa suppression $e^{-m_W r}$.

  \item \textbf{\quadStrong\ (Strong):} $m<0$, $E<0$. Confining, with three colour
        phases at $2\pi/3$ intervals. Asymptotic freedom at small $r$; linear confinement
        at large $r$. $SU(3)$ colour neutrality = closed orbit on the negative real axis.

  \item \textbf{\quadGrav\ (Gravitational):} $m>0$, $E<0$. Long-range, purely
        attractive, massless graviton on the \quadGrav--\quadEM\ boundary. Always
        attractive because there is no positive-$E$ branch in $\Delta_{\rm Grav}$.
        Coupling suppressed by $(r_{\rm P}/r)^2$.
\end{enumerate}

The quadrant boundaries (the coordinate axes) act as force-transition interfaces,
hosting the massless mediators and governing the conservation laws that apply when a
state changes its dominant interaction channel. Chapter~\ref{ch:standard-model} builds
on this picture by placing Standard Model particles—quarks, leptons, and gauge
bosons—explicitly on the complex plane and connecting their properties to the phase
structure developed here.
