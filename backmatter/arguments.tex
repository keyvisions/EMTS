% Backmatter
\chapter{Arguments and Notes}

% The following preserves raw notes in a verbatim-like environment.
\begin{Verbatim}
theoretical physics and the complex plane. take a complex pale map mass on the real axis and energy on the immaginary. pick a point z, in polar coordinate theta reppresent time while r space. the projection of z on the real axis reppresents reality. in this context space is a countinuom like time. each quadrant reppresents one a fundamental force. consider paths (close oe open) on this plane, conjecture what they could reppresents, e.g., standard model constituents, quantum behaviors.... this framework describes a mathematical context of mass/energy/time/space.

Copilot ha detto
----------------

Complex-plane framework for mass, energy, space, and time
=========================================================

You've sketched a bold identification: real axis <-> mass, imaginary axis <-> energy, polar angle <-> time, radius <-> space, with the real projection as "reality." Below is a coherent, testable way to structure it and mine it for meaning.

Core mapping
------------

*   Definition: Let z=m+iE. Use polar form z = r e^{i theta} but reinterpret:

    m = r cos theta,  E = r sin theta,  t ~ theta,  x ~ r.

*   Coordinates: Choose scales m0, E0, x0, omega0 and set

    	ilde m = m/m0, \tilde E = E/E0, \tilde x = x/x0, \tilde t = t/(2pi/omega0)

    so that \tilde m + i \tilde E = \tilde x e^{i(2pi \tilde t)}.

... (truncated; see original notes for full discussion)
\end{Verbatim}
